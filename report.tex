% Options for packages loaded elsewhere
\PassOptionsToPackage{unicode}{hyperref}
\PassOptionsToPackage{hyphens}{url}
%
\documentclass[
]{article}
\usepackage{lmodern}
\usepackage{amssymb,amsmath}
\usepackage{ifxetex,ifluatex}
\ifnum 0\ifxetex 1\fi\ifluatex 1\fi=0 % if pdftex
  \usepackage[T1]{fontenc}
  \usepackage[utf8]{inputenc}
  \usepackage{textcomp} % provide euro and other symbols
\else % if luatex or xetex
  \usepackage{unicode-math}
  \defaultfontfeatures{Scale=MatchLowercase}
  \defaultfontfeatures[\rmfamily]{Ligatures=TeX,Scale=1}
\fi
% Use upquote if available, for straight quotes in verbatim environments
\IfFileExists{upquote.sty}{\usepackage{upquote}}{}
\IfFileExists{microtype.sty}{% use microtype if available
  \usepackage[]{microtype}
  \UseMicrotypeSet[protrusion]{basicmath} % disable protrusion for tt fonts
}{}
\makeatletter
\@ifundefined{KOMAClassName}{% if non-KOMA class
  \IfFileExists{parskip.sty}{%
    \usepackage{parskip}
  }{% else
    \setlength{\parindent}{0pt}
    \setlength{\parskip}{6pt plus 2pt minus 1pt}}
}{% if KOMA class
  \KOMAoptions{parskip=half}}
\makeatother
\usepackage{xcolor}
\IfFileExists{xurl.sty}{\usepackage{xurl}}{} % add URL line breaks if available
\IfFileExists{bookmark.sty}{\usepackage{bookmark}}{\usepackage{hyperref}}
\hypersetup{
  pdftitle={Nudging for less kludges: focusing on PMD alerts as possible kludges: open, fixed or new?},
  pdfauthor={Bruno Crotman},
  hidelinks,
  pdfcreator={LaTeX via pandoc}}
\urlstyle{same} % disable monospaced font for URLs
\usepackage[margin=1in]{geometry}
\usepackage{color}
\usepackage{fancyvrb}
\newcommand{\VerbBar}{|}
\newcommand{\VERB}{\Verb[commandchars=\\\{\}]}
\DefineVerbatimEnvironment{Highlighting}{Verbatim}{commandchars=\\\{\}}
% Add ',fontsize=\small' for more characters per line
\usepackage{framed}
\definecolor{shadecolor}{RGB}{248,248,248}
\newenvironment{Shaded}{\begin{snugshade}}{\end{snugshade}}
\newcommand{\AlertTok}[1]{\textcolor[rgb]{0.94,0.16,0.16}{#1}}
\newcommand{\AnnotationTok}[1]{\textcolor[rgb]{0.56,0.35,0.01}{\textbf{\textit{#1}}}}
\newcommand{\AttributeTok}[1]{\textcolor[rgb]{0.77,0.63,0.00}{#1}}
\newcommand{\BaseNTok}[1]{\textcolor[rgb]{0.00,0.00,0.81}{#1}}
\newcommand{\BuiltInTok}[1]{#1}
\newcommand{\CharTok}[1]{\textcolor[rgb]{0.31,0.60,0.02}{#1}}
\newcommand{\CommentTok}[1]{\textcolor[rgb]{0.56,0.35,0.01}{\textit{#1}}}
\newcommand{\CommentVarTok}[1]{\textcolor[rgb]{0.56,0.35,0.01}{\textbf{\textit{#1}}}}
\newcommand{\ConstantTok}[1]{\textcolor[rgb]{0.00,0.00,0.00}{#1}}
\newcommand{\ControlFlowTok}[1]{\textcolor[rgb]{0.13,0.29,0.53}{\textbf{#1}}}
\newcommand{\DataTypeTok}[1]{\textcolor[rgb]{0.13,0.29,0.53}{#1}}
\newcommand{\DecValTok}[1]{\textcolor[rgb]{0.00,0.00,0.81}{#1}}
\newcommand{\DocumentationTok}[1]{\textcolor[rgb]{0.56,0.35,0.01}{\textbf{\textit{#1}}}}
\newcommand{\ErrorTok}[1]{\textcolor[rgb]{0.64,0.00,0.00}{\textbf{#1}}}
\newcommand{\ExtensionTok}[1]{#1}
\newcommand{\FloatTok}[1]{\textcolor[rgb]{0.00,0.00,0.81}{#1}}
\newcommand{\FunctionTok}[1]{\textcolor[rgb]{0.00,0.00,0.00}{#1}}
\newcommand{\ImportTok}[1]{#1}
\newcommand{\InformationTok}[1]{\textcolor[rgb]{0.56,0.35,0.01}{\textbf{\textit{#1}}}}
\newcommand{\KeywordTok}[1]{\textcolor[rgb]{0.13,0.29,0.53}{\textbf{#1}}}
\newcommand{\NormalTok}[1]{#1}
\newcommand{\OperatorTok}[1]{\textcolor[rgb]{0.81,0.36,0.00}{\textbf{#1}}}
\newcommand{\OtherTok}[1]{\textcolor[rgb]{0.56,0.35,0.01}{#1}}
\newcommand{\PreprocessorTok}[1]{\textcolor[rgb]{0.56,0.35,0.01}{\textit{#1}}}
\newcommand{\RegionMarkerTok}[1]{#1}
\newcommand{\SpecialCharTok}[1]{\textcolor[rgb]{0.00,0.00,0.00}{#1}}
\newcommand{\SpecialStringTok}[1]{\textcolor[rgb]{0.31,0.60,0.02}{#1}}
\newcommand{\StringTok}[1]{\textcolor[rgb]{0.31,0.60,0.02}{#1}}
\newcommand{\VariableTok}[1]{\textcolor[rgb]{0.00,0.00,0.00}{#1}}
\newcommand{\VerbatimStringTok}[1]{\textcolor[rgb]{0.31,0.60,0.02}{#1}}
\newcommand{\WarningTok}[1]{\textcolor[rgb]{0.56,0.35,0.01}{\textbf{\textit{#1}}}}
\usepackage{graphicx,grffile}
\makeatletter
\def\maxwidth{\ifdim\Gin@nat@width>\linewidth\linewidth\else\Gin@nat@width\fi}
\def\maxheight{\ifdim\Gin@nat@height>\textheight\textheight\else\Gin@nat@height\fi}
\makeatother
% Scale images if necessary, so that they will not overflow the page
% margins by default, and it is still possible to overwrite the defaults
% using explicit options in \includegraphics[width, height, ...]{}
\setkeys{Gin}{width=\maxwidth,height=\maxheight,keepaspectratio}
% Set default figure placement to htbp
\makeatletter
\def\fps@figure{htbp}
\makeatother
\setlength{\emergencystretch}{3em} % prevent overfull lines
\providecommand{\tightlist}{%
  \setlength{\itemsep}{0pt}\setlength{\parskip}{0pt}}
\setcounter{secnumdepth}{5}
\usepackage{lscape}
\newcommand{\blandscape}{\begin{landscape}}
\newcommand{\elandscape}{\end{landscape}}
\definecolor{darkred}{RGB}{150, 40, 40}
\definecolor{darkgreen}{RGB}{30, 120, 30}
\definecolor{darkorange}{RGB}{40, 40, 160}
\newcommand{\comentario}[1]{}
\usepackage{booktabs}
\usepackage{longtable}
\usepackage{array}
\usepackage{multirow}
\usepackage{wrapfig}
\usepackage{float}
\usepackage{colortbl}
\usepackage{pdflscape}
\usepackage{tabu}
\usepackage{threeparttable}
\usepackage{threeparttablex}
\usepackage[normalem]{ulem}
\usepackage{makecell}
\usepackage{xcolor}

\title{Nudging for less kludges: focusing on PMD alerts as possible kludges:
open, fixed or new?}
\author{Bruno Crotman}
\date{}

\begin{document}
\maketitle

{
\setcounter{tocdepth}{3}
\tableofcontents
}
\small

\normalsize

\section{Introduction}\label{intro}

This document is part of a research project about software degradation
caused by careless developers' behavior and about strategies to deal
with such undesired behavior. These strategies will possibly be inspired
by concepts from game theory.

We assume that software degradation can be measured by the number and
the types of \textit{kludges} made by software developers in the code. 
A kludge is code that

\begin{enumerate}
\def\labelenumi{\arabic{enumi}.}
\tightlist
\item
  Partially fixes a bug or partially implements a feature.
\end{enumerate}

\setlength{\parindent}{1.2cm}
\hangindent=1.2cm
The term partial can be understood as in \textit{partial functions}. A
partial function is undefined for some elements in the formal domain.
For instance, the square root function restricted to the integers:
\(f(25)\) is defined, but \(f(26)\) is undefined. In terms of features,
we can think about a developer calculating the point on which two lines
cross and neglecting the case of parallel lines

\begin{enumerate}
\def\labelenumi{\arabic{enumi}.}
\setcounter{enumi}{1}
\tightlist
\item
  The developer knows that the code is only a partial solution, with
  high probability. \footnote{We need to study technical debt papers
  to enrich the conceptual background.}
\end{enumerate}

This project aims to study how software projects evolve in terms of number
and kinds of kludges. So far, we are trying to identify kludges by looking 
at alerts generated by the PMD source code analyzer. PMD is static source 
code analyzer that is commonly used to find possible programming flaws. 
These are the planned steps for this research project:

%
% MARCIO: justificar porque decidiu pelo PMD.
%

\begin{itemize}
\item
  Confirm the assumption that the frequency of PMD alerts is an accurate
  measure of the prevalence of kludges;
  
\item
  Confirm the assumption that kludges harm software development;

\item
  Confirm the assumption that there is a game in which, in Nash
  equilibrium, a developer chooses a strategy in which he gets personal
  benefits while causing harm to the project, by making kludges;

\item
  If all these assumptions are true, use mechanism design to devise how
  we can change the environment in a way that developers do not choose
  to make so much kludge, increasing the quality of the project in the
  long run.

\item
  Implement this mechanism building a plugin for a prominent CI tool,
  such as Travis or Jenkins or GitLab.
\end{itemize}


% ========================================================= \% 
% PMD SOURCE CODE ANALYZER 
% %=========================================================

\section{The PMD Source Code Analyzer}\label{pmd}

We use PMD to list the alerts that represent \textit{possible kludges} in a source code. PMD receives a source code as input and generates a list of bad programming practices contained in the code, i.e., the alerts. The process we follow to generate the alerts using PMD source code analyzer is discussed in Section \ref{history}.

%\item To create an Abstract Syntax Tree (AST) from a source code with selected nodes. This will help us in the algorithm described in Section \ref{alg}. The creation of the AST using PMD is described in Section \ref{ast}.

PMD traverses the AST of a source code searching for violations of rules
which are configured by the user. PMD comes with a default rule set for
the Java programming language. The default rule set finds common
programming flaws such as unused variables, empty catch blocks,
unnecessary object creation, and so forth. It is possible to configure a
different set of rules by creating a custom XML file. In Figure
\ref{simple_code}, we can see a simple code and the alerts
that were generated by the default rule set of PMD alerts.

\vspace{16px}
\scriptsize
\begin{Shaded}
\begin{Highlighting}[]
\CommentTok{/*  1-                                   */}\KeywordTok{package}\ImportTok{ pack_x;}
\CommentTok{/*  2-                                   */}  
\CommentTok{/*  3-                                   */}\KeywordTok{import}\ImportTok{ importX.function;}
\CommentTok{/*  4-                                   */}
\CommentTok{/*  5-                                   */}\KeywordTok{class}\NormalTok{ ClassX }\KeywordTok{extends}\NormalTok{ ClassY }\KeywordTok{implements}\NormalTok{ InterfX \{}
\CommentTok{/*  6-                                   */}    \KeywordTok{private} \DataTypeTok{long}\NormalTok{ fieldX;}
\CommentTok{/*  7-                                   */}    
\CommentTok{/*  8-                                   */}    \FunctionTok{ClassX}\NormalTok{(}\DataTypeTok{int}\NormalTok{ paramX, }\DataTypeTok{double}\NormalTok{ paramY) \{      }
\CommentTok{/*  9-                                   */}        \DataTypeTok{int}\NormalTok{ varX = }\FunctionTok{function}\NormalTok{(paramX, paramY);     }
\CommentTok{/* 10-                                   */}        \KeywordTok{if}\NormalTok{ (varX == }\DecValTok{0}\NormalTok{)}
\CommentTok{/* 11-ControlStatementBraces             */}            \KeywordTok{this}\NormalTok{.}\FunctionTok{fieldX}\NormalTok{ = }\DecValTok{1}\NormalTok{;}
\CommentTok{/* 12-                                   */}        \KeywordTok{else}\NormalTok{\{}
\CommentTok{/* 13-                                   */}            \KeywordTok{this}\NormalTok{.}\FunctionTok{fieldX}\NormalTok{ = }\DecValTok{0}\NormalTok{;}
\CommentTok{/* 14-                                   */}\NormalTok{     \}}
\CommentTok{/* 15-                                   */}\NormalTok{    \}}
\CommentTok{/* 16-                                   */}    \AttributeTok{@Override}
\CommentTok{/* 17-                                   */}    \KeywordTok{public} \DataTypeTok{int} \FunctionTok{methodX}\NormalTok{(}\DataTypeTok{int}\NormalTok{ paramW, }\BuiltInTok{Boolean}\NormalTok{ paramZ)}
\CommentTok{/* 18-                                   */}\NormalTok{    \{}
\CommentTok{/* 19-                                   */}        \KeywordTok{if}\NormalTok{ (paramZ)}
\CommentTok{/* 20-ControlStatementBraces             */}\NormalTok{            fieldX = paramW;}
\CommentTok{/* 21-                                   */}        \KeywordTok{else}\NormalTok{\{}
\CommentTok{/* 22-                                   */}\NormalTok{            fieldX = }\DecValTok{0}\NormalTok{;}
\CommentTok{/* 23-                                   */}\NormalTok{     \}}
\CommentTok{/* 24-                                   */}        \KeywordTok{return}\NormalTok{ paramW + }\KeywordTok{this}\NormalTok{.}\FunctionTok{fieldX}\NormalTok{;}
\CommentTok{/* 25-                                   */}\NormalTok{     \}}
\CommentTok{/* 26-                                   */}\NormalTok{\}  }
\end{Highlighting}
\end{Shaded}

\normalsize


% usar uma siglas para os alertas para que em poucos caracteres possamos identificá-los

\begin{figure}
\centering
\includegraphics{figures/fake.png}
\caption{Simple code with its alerts \label{simple_code}}
\end{figure}

% tirar a figura fake

\subsection{Using PMD to capture the history of alerts}\label{history}

To evaluate how the number of alerts evolved throughout the history of a
software project, we must be able to analyze two different versions of a
source code (an old and a new version) and categorize each alert
contained in the code as either \textbf{new}, \textbf{fixed} or
\textbf{open}.

We define a PMD alert generated for the old version as either
\textbf{open} or \textbf{fixed} in the new version. An \textbf{open}
alert remains in the new version of the code. A \textbf{fixed} alert
does not exist in the new version.

A PMD alert generated for the new version is either \textbf{open} or
\textbf{new}. An \textbf{open} alert indicates that the same alert was
identified in the old version of the source code. A \textbf{new} alert
implies that the same alert cannot be identified in the old version.

The intersection between \textbf{fixed} alerts, \textbf{new} alerts and
\textbf{open} alerts is empty. The alerts identified as \textbf{open}
are equivalent in both new and old versions. To decide whether an alert
is \textbf{open}, \textbf{fixed} or \textbf{new}, one has to identify if
this alert in the old version is equivalent to its occurrence in the new
version. This document describes an algorithm to make this classification. 
At this point, we use the default rule set to generate the alerts.

%\subsection{Using PMD to generate an Abstract Syntax Tree}\label{ast}

%The AST is used to understand the location of an alert in a version of a
%code. We use this information in the algorithm described in Section
%\ref{alg}.

%PMD traverses the source code visiting many different kinds of elements
%comprising the AST. If we build our own rules, aimed to capture only
%some kinds of elements, we will generate list of ``alerts'' that will
%contain all the elements of the chosen kinds contained in the AST.

\subsection{Using PMD to generate an Abstract Syntax Tree }\label{ast}
%olhar a frase abaixo
%o objetivo da seção é explicar que o mapeamento das linhas de código é falho então uso a ast pra me ajudar e eu seleciono alguns tipos de nós da ast
The AST is used, in conjunction with a mapping of the lines of code
in the new and old versions, to understand the location of an alert in
a version of a code. The tree helps us in making matches between the
old and the new versions that are not exact. Rather then betting on a
match that depends solely on the location of the node where the alert
is, we can get the information that the alerts are not in the same line
but belong to methods that are in the same line, for instance.

%classorinterfacedeclaration e classorinterfacetype
%enxugar a lista de tipos de nós


%
% MARCIO: não entendi esta estória de métodos na mesma linha. Você quer dizer que os
% alertas não estão na mesma linha mas pertencem ao mesmo método?
%


%
%Dizer que o PMD neste caso aí devolve uma lista de nós que não estão relacionados e eles precisam ser relacionados em uma árvore
%

PMD traverses the source code visiting many different kinds of elements. 
%If we build our own rules to capture some kinds of elements, we will 
%generate list of ``alerts'' that will contain all the elements of the
%chosen kinds contained in the AST. 
We do not use all the types of nodes 
recognized by PMD Alert to generate the AST because there are many kinds 
of nodes and some of them are redundant for our purposes. If we used all
the kinds of nodes, we would end up with a tree that would not add value
to our analysis but would add complexity to our algorithm. The kinds of 
elements that were selected are listed below:

%
% MARCIO: porque são redundantes? Qual é o propósito - reconhecer que um alerta
% pertence ao mesmo método?
%

\begin{itemize}

\item \textbf{Annotation}: a syntatic metadata added to a source code;

\item \textbf{Block}: a block of statements enclosed by braces;

\item \textbf{ClassOrInterfaceBody}: the body of an interface or a class, excluding the declaration;

\item \textbf{ClassOrInterfaceDeclaration}: class or interface, including the declaration and the body;

\item \textbf{ClassOrInterfaceType}: declaration of a type;

\item \textbf{CompilationUnit}: the root of an AST tree;

\item \textbf{ConstructorDecaration}: class or interface, including the declaration and the body;

\item \textbf{ExtendsList}: list of extensions of a class;

\item \textbf{FieldDeclaration}: field declaration, including the type, name and possible assignment;

\item \textbf{FormalParameter}: a parameter of a method or constructor;

\item \textbf{FormalParameters}:  list of formal parameters of a method or constructor;

\item \textbf{IfStatement}: an if statements including its blocks and condition;

\item \textbf{ImplementsList}: list of implementations of an interface;

\item \textbf{Import}: a package imported;

\item \textbf{Method}: a method, including body and declaration;

\item \textbf{Name}: a named value, like a variable, a class or a package referenced in the code.

\item \textbf{Package}: the indication of the package to which the compilation unit belongs;

\item \textbf{Statement}: any statement, like an if statement or an assignment;

\item \textbf{TypeDeclaration}: any type declaration;

\item \textbf{VariableId}: the name of a variable in a variable declaration.
\end{itemize}

We follow three steps to recreate the AST:

\begin{enumerate}
\def\labelenumi{\arabic{enumi}.}
\item
  Link each element \(a\) to the set of elements \(X\) that are fully
  contained between the begin line / begin column and end line / end
  column of element \(a\). We can construct a directed graph in which
  the elements are the nodes and the links described are the edges. This
  is not a tree yet, because each node will have edges directed to all
  its descendants and not only its children in the AST.

\item
  Sort the nodes in the decreasing order of its number of children. The
  objective is to establish that, in a search through this graph, the
  first child chosen will be the one that is a child in the AST, and not
  only a descendant.

\item
  Proceed a deep-first search starting from the compilation unit node.
\end{enumerate}

% MARCIO: eu não entendi o objetivo do algoritmo acima. Você quer montar a AST somente \% com os tipos de nós que listou acima? Por quê? Justifique. Depois disso, quer recriar \% o código-fonte a partir desta AST parcial? Porque precisa disso? \%

% BRUNO: A quantidade de tipos de nós é gigantesca. A árvore fica gigantesca e muito redundante. A ideia é permitir matches que não sejam puramente baseados \% na localização das linhas. Expliquei no texto. Às vezes pode não bater exatamente a localização, mas o nó método pai pode bater, portanto, podemos \% considerar que os nós estão no mesmo método e se eles forem de mesmo tipo, com o mesmo alerta, debaixo do mesmo método\ldots{} se tiver muita evidência de que \% é a mesma coisa, podemos considerar que é. Aquela história: tem focinho de porco, orelha de porco, pé de porco. É porco? Só não sei como escrever isso \% para um inglês..

% MARCIO: o problema é justificar esta redundância. Ela exige que você assuma um nível de corte na localização de um alerta - ao invés de associá-lo a um comando, você o associa a um método? Isto não está claro para mim.

\section{Research questions}
\label{as_whole}

%em vez de evento, usar transição entre versões, já que não estamos mais pensando em commits contíguos 

\noindent
\textbf{RQ: Is the frequency of PMD Alerts an accurate measure of the prevalence of kludges?}
%\label{PMD_Kludge}

We have to identify events with an intense introduction of PMD Alert. To
do this we have to be able to categorize open, new and fixed PMD Alerts.

%
% MARCIO: o que é um evento? Uma edição?
%

%acertar a ordem da frase
For each pair of an old and a new version of a source code, we can
measure the intensity of possible kludge introduction occurred. This
evidence of possible kludge introduction must be normalized by the size
of the change in source code between the two versions. The measure may
follow a formula like:

\[ \frac{\#NewAlerts - \#FixedAlerts}{Change}    \]

or

\[ \frac{\#NewAlerts - \#FixedAlerts}{\#NewAlerts + \#FixedAlerts}    \]


%Olhar a Jaccard, que parece a fórmula da proporção 

%
% MARCIO: qual das duas? existe uma vantagem em usar uma ou outra?
%

With these events identified, including their location in the code and
their position in the time line of commits, we can measure the
correlation of these events with some other evidence of kludge. In
Section \ref{results} we calulate these metrics for

% % MARCIO: o que seria ``their moment''? Uma versão em que foram introduzidos? \% se você está preocupado com a introdução de issues, porque está tratando \% os issues resolvidos (no numerador) nas fórmulas acima? %

% \%BRUNO: Na primerira fórmula eu vou usar uma medida baseada no número de linhas, na segunda realmente eu uso o número total de alertas. É uma proporção de \% de quantos são os novos alertas em relação ao núymero total de alertas. O número total de alertas fechados e abertos serviria como ``tamanho da mudança''. É ruim? \%

An evidence could be churns around the location of the event in
subsequent commits. And this would be related with the statement 1 of
the definition of kludge in Section \ref{as_whole}.

Other possible evidence could be survey and bag of words mining of
issue, mailing lists, and commit messages showing evidence of kludge
game. This could be related more with the Statement 2 of the definition
of kludge in Section \ref{as_whole}.

\vspace{16px}
\noindent
\textbf{RQ: Do kludges harm software development?} \label{kludge_harm}

We need some way to measure degradation after a heavy introduction of
kludges. A drop in the popularity may not be a proper evidence. The
increment in the number of issues and bug fixed nor necessarily
represent a degradation. Churn could be used here.

% ========================================================= 
% % ALGORITHMS \% \%
%=========================================================

\section{Algorithm to categorize alerts}\label{alg}

This Section discusses the algorithm to categorize alerts, which uses
the AST to create features that help to infer if two alerts in different
versions must be considered the same. This algorithm does not return a
single categorical answer for each alert (new, open or fixed), but
features that must be evaluated by some heuristic or possibly a
statistical learning tool.

%In Section \ref{source_used} the source code used in this description is
%shown. Section \ref{map} describes how the source code lines are mapped.
%In the Section \ref{feature_creation}, we show how the features are
%created. Even though it is not possible to be sure if a pair alerts in
%different versions are same alert, we think that these features can
%provide some clues. Sections from \ref{example_rename_method} to \ref{example_editing_line} use this %algorithm in some examples. 
%The heuristic we used for this work is described in
%Section \ref{heuristic}.

% \% MARCIO: retorna features? o que são estas features? %

% \% BRUNO: o pessoal de aprendizado estatístico que chama de feature cada tipo de dado usado como entrada nos modelos deles. Tem até uma área de feature \% engineering, que serve pra gerar, a partir dos dados brutos, esses dados de entrada que facilitam a vida do modelo. Por exemplo, ao invés de enfiar num \% modelo uma imagem de um exame, vc pode fazer primeiro um tratamento que extrai o diâmetro médio das células que ele encontra na imagem. Seria o paralelo \% de em vez de meter no nosso modelo (que no caso é uma heurística) o código-fonte todo, a gente meter essas features pra que seja mais fácil definir a \% classificação do alerta. Vc acha que vale a pena eu escrever essas coisas no texto?

% MARCIO: escreve isso ...

\newpage
\begin{landscape}

\subsection{An illustrative example}\label{source_used}

In this Section, we will consider the old and new version of a source
code as presented in Figure \ref{old_and_new_figure}. In the new version, the alert generated in
the line 11 of the old version was fixed.

\vspace{16px}
\scriptsize

\begin{Shaded}
\begin{Highlighting}[]
\CommentTok{/*  1-                 */}
\KeywordTok{package}\ImportTok{ pack_x;                                                /*  1-                 */package pack_x;}                                                
\CommentTok{/*  2-                 */}\CommentTok{/*  2-                 */}                                                               
\CommentTok{/*  3-                 */}\KeywordTok{import}\ImportTok{ importX.function;                                       /*  3-                 */import importX.function;}
\CommentTok{/*  4-                 */}\CommentTok{/*  4-                 */}
\CommentTok{/*  5-                 */}\KeywordTok{class}\NormalTok{ ClassX }\KeywordTok{extends}\NormalTok{ ClassY }\KeywordTok{implements}\NormalTok{ InterfX \{               }\CommentTok{/*  5-                 */}\KeywordTok{class}\NormalTok{ ClassX }\KeywordTok{extends}\NormalTok{ ClassY }\KeywordTok{implements}\NormalTok{ InterfX \{               }
\CommentTok{/*  6-                 */}    \KeywordTok{private} \DataTypeTok{long}\NormalTok{ fieldX;                                       }\CommentTok{/*  6-                 */}    \KeywordTok{private} \DataTypeTok{long}\NormalTok{ fieldX;                                       }
\CommentTok{/*  7-                 */}                                                               \CommentTok{/*  7-                 */}                                                               
\CommentTok{/*  8-                 */}    \FunctionTok{ClassX}\NormalTok{(}\DataTypeTok{int}\NormalTok{ paramX, }\DataTypeTok{double}\NormalTok{ paramY) \{                        }\CommentTok{/*  8-                 */}    \FunctionTok{ClassX}\NormalTok{(}\DataTypeTok{int}\NormalTok{ paramX, }\DataTypeTok{double}\NormalTok{ paramY) \{ }
\CommentTok{/*  9-                 */}        \DataTypeTok{int}\NormalTok{ varX = }\FunctionTok{function}\NormalTok{(paramX, paramY);                   }\CommentTok{/*  9-                 */}        \DataTypeTok{int}\NormalTok{ varX = }\FunctionTok{function}\NormalTok{(paramX, paramY);                           }
\CommentTok{/* 10-                 */}        \KeywordTok{if}\NormalTok{ (varX == }\DecValTok{0}\NormalTok{)                                         }\CommentTok{/* 10-                 */}        \KeywordTok{if}\NormalTok{ (varX == }\DecValTok{0}\NormalTok{)                                         }
\CommentTok{/*   -                 *//*XXXXXXXXXXXXXXXXXXXXXXXXXXXXXXXXXXXXXX*/}                     \CommentTok{/* 11-                 */}\NormalTok{        \{                                                      }
\CommentTok{/* 11-ControlStateme   */}            \KeywordTok{this}\NormalTok{.}\FunctionTok{fieldX}\NormalTok{ = }\DecValTok{1}\NormalTok{;                                   }\CommentTok{/* 12-                 */}            \KeywordTok{this}\NormalTok{.}\FunctionTok{fieldX}\NormalTok{ = }\DecValTok{1}\NormalTok{;                                   }
\CommentTok{/*   -                 *//*XXXXXXXXXXXXXXXXXXXXXXXXXXXXXXXXXXXXXX*/}                     \CommentTok{/* 13-                 */}\NormalTok{        \}                                                                }
\CommentTok{/* 12-                 */}        \KeywordTok{else}\NormalTok{\{                                                  }\CommentTok{/* 14-                 */}        \KeywordTok{else}\NormalTok{\{                                                  }
\CommentTok{/* 13-                 */}            \KeywordTok{this}\NormalTok{.}\FunctionTok{fieldX}\NormalTok{ = }\DecValTok{0}\NormalTok{;                                   }\CommentTok{/* 15-                 */}            \KeywordTok{this}\NormalTok{.}\FunctionTok{fieldX}\NormalTok{ = }\DecValTok{0}\NormalTok{;                                   }
\CommentTok{/* 14-                 */}\NormalTok{       \}                                                       }\CommentTok{/* 16-                 */}\NormalTok{        \}                                                              }
\CommentTok{/* 15-                 */}\NormalTok{    \}                                                          }\CommentTok{/* 17-                 */}\NormalTok{    \}                                           }
\CommentTok{/* 16-                 */}    \AttributeTok{@Override}                                                  \CommentTok{/* 18-                 */}    \AttributeTok{@Override}                                                  
\CommentTok{/* 17-                 */}    \KeywordTok{public} \DataTypeTok{int} \FunctionTok{methodX}\NormalTok{(}\DataTypeTok{int}\NormalTok{ paramW, }\BuiltInTok{Boolean}\NormalTok{ paramZ)             }\CommentTok{/* 19-                 */}    \KeywordTok{public} \DataTypeTok{int} \FunctionTok{methodX}\NormalTok{(}\DataTypeTok{int}\NormalTok{ paramW, }\BuiltInTok{Boolean}\NormalTok{ paramZ)             }
\CommentTok{/* 18-                 */}\NormalTok{    \{                                                          }\CommentTok{/* 20-                 */}\NormalTok{    \{                                                          }
\CommentTok{/* 19-                 */}        \KeywordTok{if}\NormalTok{ (paramZ)                                            }\CommentTok{/* 21-                 */}        \KeywordTok{if}\NormalTok{ (paramZ)                                            }
\CommentTok{/* 20-ControlStateme   */}\NormalTok{            fieldX = paramW;                                   }\CommentTok{/* 22-ControlStateme   */}\NormalTok{            fieldX = paramW;                                   }
\CommentTok{/* 21-                 */}        \KeywordTok{else}\NormalTok{\{                                                  }\CommentTok{/* 23-                 */}        \KeywordTok{else}\NormalTok{\{                                                  }
\CommentTok{/* 22-                 */}\NormalTok{            fieldX = }\DecValTok{0}\NormalTok{;                                        }\CommentTok{/* 24-                 */}\NormalTok{            fieldX = }\DecValTok{0}\NormalTok{;                                        }
\CommentTok{/* 23-                 */}\NormalTok{       \}                                                       }\CommentTok{/* 25-                 */}\NormalTok{        \}                                                              }
\CommentTok{/* 24-                 */}        \KeywordTok{return}\NormalTok{ paramW + }\KeywordTok{this}\NormalTok{.}\FunctionTok{fieldX}\NormalTok{;                           }\CommentTok{/* 26-                 */}        \KeywordTok{return}\NormalTok{ paramW + }\KeywordTok{this}\NormalTok{.}\FunctionTok{fieldX}\NormalTok{;          }
\CommentTok{/* 25-                 */}\NormalTok{     \}                                                         }\CommentTok{/* 27-                 */}\NormalTok{     \}                                                         }
\CommentTok{/* 26-                 */}\NormalTok{\}                                                              }\CommentTok{/* 28-                 */}\NormalTok{\}                                                              }
\end{Highlighting}
\end{Shaded}

\normalsize

\begin{figure}
\centering
\includegraphics{figures/fake.png}
\caption{An illustrative example of a source code subject to the proposed algorithm.}\label{old_and_new_figure}
\end{figure}

\end{landscape}

\newpage

\subsection{Using information from git diff, create a relation between the lines}\label{map}

For each difference stated in the output (the sections of the diff file
starting with ``@@''), there is an indication of the number of lines
removed from the old version and the number of lines added to the new
one. The line in which the lines are removed from the old version and
the line at which the lines are added is indicated, too.\\
By using this information we create a relation between the lines of the
old version and the equivalent lines in the new version. For the new and
old versions presented in Section \ref{source_used}, the relation is shown in
Table \ref{table_map}.

\small

\begin{table}

\caption{\label{tab:showing map}Relation between lines of the old version and lines of the new version\label{table_map}}
\centering
\resizebox{\linewidth}{!}{
\begin{tabular}[t]{l|l|l|l|l|l|l|l|l|l|l|l|l|l|l}
\hline
1 & 2 & 3- & -8 & 9 & 10 & \textcolor{white}{7} & 11 & \textcolor{white}{9} & 12 & 13 & 14- & -24 & 25 & 26\\
\hline
1 & 2 & 3- & -8 & 9 & 10 & 11 & 12 & 13 & 14 & 15 & 16- & -26 & 27 & 28\\
\hline
\end{tabular}}
\end{table}

\normalsize

\subsection{Feature engineering} \label{feature_creation}

In Figure \ref{AST_compare_id_alerts} we can see the ASTs for the old
and the new versions. We can see the kind of nodes and the PMD alert if
there is one. In this figures, the numbers in the nodes are meaningless
and are presented only for reference.

\small

\begin{figure}[H]
\includegraphics[width=.8\linewidth]{report_files/figure-latex/unnamed-chunk-2-1} \caption{Abstract Syntax Trees. New and old versions, with alerts \label{AST_compare_id_alerts}}\label{fig:unnamed-chunk-2}
\end{figure}

\normalsize

\begin{itemize}
\item
  Considering the relation between the lines of the two versions constructed as we saw in Section \ref{map}, the nodes in both trees begin and end in related lines;

\item
  The nodes are of the same kind.
\end{itemize}

\small

\begin{figure}[H]
\includegraphics[width=.8\linewidth]{report_files/figure-latex/unnamed-chunk-3-1} \caption{Abstract Syntax Tree. Nodes with the same number are equivalent \label{AST_with_alerts}}\label{fig:unnamed-chunk-3}
\end{figure}

\normalsize

The path between the alert and the root of the AST can be seen in
\ref{AST_alert_1}. Comparing the path of different alerts is possible to
determine if the nodes belong to the same method, for instance.

\small

\begin{figure}[H]
\includegraphics[width=1\linewidth]{report_files/figure-latex/unnamed-chunk-4-1} \caption{Abstract Syntax Tree \label{AST_alert_1}}\label{fig:unnamed-chunk-4}
\end{figure}

\normalsize

The algorithm generates a set of features for each pair of alerts
\((n,o)\) with one element \(n\) coming from the old version and one
element \(o\) coming from the new version. The features do not lead to a
direct conclusion. It´s necessary to create a heuristic or statistical
learning algorithm that will decide the final verdict based on the
features.

We propose the following list of features:

\begin{itemize}
\item
  Same Rule: a boolean indicator that tells if the alerts are of the
  same type
\item
  Same Group ID: a boolean indicator that tells if the alerts are
  equivalent as defined in Figure \ref{AST_groups}
\item
  Same Method Group ID: a boolean indicator that tells if the alerts
  belong to the same method. We know the alert's method following the
  path from the alert´s node to the root. The first node of the kind
  ``method'' found in this path defines the alert's method. If this is
  the same for \(o\) and for \(n\), then they belong to the same method.
\item
  Same Method Name: a boolean indicator that tells if the alerts were
  found in a method with the same name.
\item
  Same Block: a boolean indicator that shows if the \(o\) and \(n\)
  belong to the same block. It is defined the same way the ``Same
  method'' indicator is defined.
\item
  Same Code: a boolean indicator that shows the nodes that generate the
  alert have the same programming code.
\item
  Same Method Code: a boolean indicator that shows that the methods that
  contain the nodes that generate the alert have the same programming
  code.
\item
  Line distance: \(o\) and \(n\) have a begin line \(b(o)\) and \(b(n)\)
  and an end line \(e(n)\) and \(e(n)\). Line distance is
  \(abs(mean(b(o), e(o)) - mean(b(n), e(n)))\)
\item
  Normalized line distance (block size): this is the line distance but
  normalized by the size of the last common node.
\item
  Normalized line distance (method size): this is the line distance but
  normalized by the size of the last common method (if there is no
  common method, it´s normalized by the side of the compilation unit).
\item
  Normalized line distance (compilation unit size): this is the line
  distance but normalized by the size of the compilation unit.
\end{itemize}

Table \ref{table_features} shows the combinations \((n,o)\) in the
example. There are \(2 \cdot 1 = 2\) combinations whereas we have two
alerts in the old version and one alert in the new one.

\small

\begin{table}[!h]

\caption{\label{tab:unnamed-chunk-5}Resulting features\label{table_features} }
\centering
\begin{tabular}[t]{l|l|l}
\hline
Alert combination & Feature & Value\\
\hline
\rowcolor{gray!6}   & Same Rule & TRUE\\

 & Same Group ID & TRUE\\

\rowcolor{gray!6}   & Same Method Group ID & TRUE\\

 & Same Method Name & TRUE\\

\rowcolor{gray!6}   & Same Block & TRUE\\

 & Same Code & TRUE\\

\rowcolor{gray!6}   & Same Method Code & TRUE\\

 & Line Distance & 0.00\\

\rowcolor{gray!6}   & Line Distance Normalized by Block Size & 0.00\\

 & Line Distance Normalized by Method Size & 0.00\\

\multirow[t]{-11}{*}{\raggedright\arraybackslash Line (Old version):20, Line (New version):22} & Line Distance Normalized by Compilation Unit Size & 0.00\\
\hline
\end{tabular}
\end{table}

\normalsize

\small

\normalsize

\newpage

\begin{landscape}

\subsection{Example: Renaming method} \label{example_rename_method}

In this example, the new and old versions have only one alert. The
method in which the alert happens is renamed from MethodX to methodZ.

\scriptsize

\begin{Shaded}
\begin{Highlighting}[]
\CommentTok{/*  1-                 */}\KeywordTok{package}\ImportTok{ pack_x;                                                /*  1-                 */package pack_x;}                                                
\CommentTok{/*  2-                 */}                                                               \CommentTok{/*  2-                 */}                                                               
\CommentTok{/*  3-                 */}\KeywordTok{import}\ImportTok{ importX.function;                                       /*  3-                 */import importX.function;}                                       
\CommentTok{/*  4-                 */}                                                               \CommentTok{/*  4-                 */}                                                               
\CommentTok{/*  5-                 */}\KeywordTok{class}\NormalTok{ ClassX }\KeywordTok{extends}\NormalTok{ ClassY }\KeywordTok{implements}\NormalTok{ InterfX \{               }\CommentTok{/*  5-                 */}\KeywordTok{class}\NormalTok{ ClassX }\KeywordTok{extends}\NormalTok{ ClassY }\KeywordTok{implements}\NormalTok{ InterfX \{               }
\CommentTok{/*  6-                 */}    \KeywordTok{private} \DataTypeTok{long}\NormalTok{ fieldX;                                       }\CommentTok{/*  6-                 */}    \KeywordTok{private} \DataTypeTok{long}\NormalTok{ fieldX;                                       }
\CommentTok{/*  7-                 */}                                                               \CommentTok{/*  7-                 */}                                                               
\CommentTok{/*  8-                 */}    \FunctionTok{ClassX}\NormalTok{(}\DataTypeTok{int}\NormalTok{ paramX, }\DataTypeTok{double}\NormalTok{ paramY) \{                                }\CommentTok{/*  8-                 */}    \FunctionTok{ClassX}\NormalTok{(}\DataTypeTok{int}\NormalTok{ paramX, }\DataTypeTok{double}\NormalTok{ paramY) \{                                }
\CommentTok{/*  9-                 */}        \DataTypeTok{int}\NormalTok{ varX = }\FunctionTok{function}\NormalTok{(paramX, paramY);                          }\CommentTok{/*  9-                 */}        \DataTypeTok{int}\NormalTok{ varX = }\FunctionTok{function}\NormalTok{(paramX, paramY);                           }
\CommentTok{/*   -                 *//*XXXXXXXXXXXXXXXXXXXXXXXXXXXXXXXXXXXXXX*/}                     \CommentTok{/* 10-                 */}        \KeywordTok{if}\NormalTok{ (varX == }\DecValTok{0}\NormalTok{)                                         }
\CommentTok{/* 10-                 */}        \KeywordTok{if}\NormalTok{ (varX == }\DecValTok{0}\NormalTok{)\{                                        }\CommentTok{/*   -                 *//*XXXXXXXXXXXXXXXXXXXXXXXXXXXXXXXXXXXXXX*/}                     
\CommentTok{/*   -                 *//*XXXXXXXXXXXXXXXXXXXXXXXXXXXXXXXXXXXXXX*/}                     \CommentTok{/* 11-                 */}\NormalTok{        \{                                                      }
\CommentTok{/* 11-                 */}            \KeywordTok{this}\NormalTok{.}\FunctionTok{fieldX}\NormalTok{ = }\DecValTok{1}\NormalTok{;                                   }\CommentTok{/* 12-                 */}            \KeywordTok{this}\NormalTok{.}\FunctionTok{fieldX}\NormalTok{ = }\DecValTok{1}\NormalTok{;                                   }
\CommentTok{/* 12-                 */}\NormalTok{        \}                                                      }\CommentTok{/*   -                 *//*XXXXXXXXXXXXXXXXXXXXXXXXXXXXXXXXXXXXXX*/}                     
\CommentTok{/*   -                 *//*XXXXXXXXXXXXXXXXXXXXXXXXXXXXXXXXXXXXXX*/}                     \CommentTok{/* 13-                 */}\NormalTok{        \}                                                                }
\CommentTok{/* 13-                 */}        \KeywordTok{else}\NormalTok{\{                                                  }\CommentTok{/* 14-                 */}        \KeywordTok{else}\NormalTok{\{                                                  }
\CommentTok{/* 14-                 */}            \KeywordTok{this}\NormalTok{.}\FunctionTok{fieldX}\NormalTok{ = }\DecValTok{0}\NormalTok{;                                   }\CommentTok{/* 15-                 */}            \KeywordTok{this}\NormalTok{.}\FunctionTok{fieldX}\NormalTok{ = }\DecValTok{0}\NormalTok{;                                   }
\CommentTok{/* 15-                 */}\NormalTok{        \}                                                      }\CommentTok{/* 16-                 */}\NormalTok{        \}                                                      }
\CommentTok{/* 16-                 */}\NormalTok{    \}                                                          }\CommentTok{/* 17-                 */}\NormalTok{    \}                                                          }
\CommentTok{/* 17-                 */}    \AttributeTok{@Override}                                                  \CommentTok{/* 18-                 */}    \AttributeTok{@Override}                                                  
\CommentTok{/* 18-                 */}    \KeywordTok{public} \DataTypeTok{int} \FunctionTok{methodX}\NormalTok{(}\DataTypeTok{int}\NormalTok{ paramW, }\BuiltInTok{Boolean}\NormalTok{ paramZ)             }\CommentTok{/*   -                 *//*XXXXXXXXXXXXXXXXXXXXXXXXXXXXXXXXXXXXXX*/}                     
\CommentTok{/*   -                 *//*XXXXXXXXXXXXXXXXXXXXXXXXXXXXXXXXXXXXXX*/}                     \CommentTok{/* 19-                 */}    \KeywordTok{public} \DataTypeTok{int} \FunctionTok{methodZ}\NormalTok{(}\DataTypeTok{int}\NormalTok{ paramW, }\BuiltInTok{Boolean}\NormalTok{ paramZ)             }
\CommentTok{/* 19-                 */}\NormalTok{    \{                                                          }\CommentTok{/* 20-                 */}\NormalTok{    \{                                                          }
\CommentTok{/* 20-                 */}        \KeywordTok{if}\NormalTok{ (paramZ)                                            }\CommentTok{/* 21-                 */}        \KeywordTok{if}\NormalTok{ (paramZ)                                            }
\CommentTok{/* 21-ControlStateme   */}\NormalTok{            fieldX = paramW;                                   }\CommentTok{/* 22-ControlStateme   */}\NormalTok{            fieldX = paramW;                                   }
\CommentTok{/* 22-                 */}        \KeywordTok{else}\NormalTok{\{                                                  }\CommentTok{/* 23-                 */}        \KeywordTok{else}\NormalTok{\{                                                  }
\CommentTok{/* 23-                 */}\NormalTok{            fieldX = }\DecValTok{0}\NormalTok{;                                        }\CommentTok{/* 24-                 */}\NormalTok{            fieldX = }\DecValTok{0}\NormalTok{;                                        }
\CommentTok{/* 24-                 */}\NormalTok{        \}                                                      }\CommentTok{/* 25-                 */}\NormalTok{        \}                                                      }
\CommentTok{/* 25-                 */}        \KeywordTok{return}\NormalTok{ paramW + }\KeywordTok{this}\NormalTok{.}\FunctionTok{fieldX}\NormalTok{;                           }\CommentTok{/* 26-                 */}        \KeywordTok{return}\NormalTok{ paramW + }\KeywordTok{this}\NormalTok{.}\FunctionTok{fieldX}\NormalTok{;                           }
\CommentTok{/* 26-                 */}\NormalTok{     \}                                                         }\CommentTok{/* 27-                 */}\NormalTok{     \}                                                         }
\CommentTok{/* 27-                 */}\NormalTok{\}                                                              }\CommentTok{/* 28-                 */}\NormalTok{\}                                                              }
\end{Highlighting}
\end{Shaded}

\normalsize

\begin{figure}
\centering
\includegraphics{figures/fake.png}
\caption{Comparison between old and new version
\label{comparison_rename}}
\end{figure}

\end{landscape}

\newpage

In the Table \ref{features_rename} we can see that the features ``Same
Method Name'' and ``Same Method Group ID'' are now FALSE.

\small

\begin{table}[!h]

\caption{\label{tab:unnamed-chunk-7}Resulting features: rename method example \label{features_rename} }
\centering
\begin{tabular}[t]{l|l|l}
\hline
Alert combination & Feature & Value\\
\hline
\rowcolor{gray!6}   & Same Rule & TRUE\\

 & Same Group ID & TRUE\\

\rowcolor{gray!6}   & Same Method Group ID & FALSE\\

 & Same Method Name & FALSE\\

\rowcolor{gray!6}   & Same Block & TRUE\\

 & Same Code & TRUE\\

\rowcolor{gray!6}   & Same Method Code & FALSE\\

 & Line Distance & 0.00\\

\rowcolor{gray!6}   & Line Distance Normalized by Block Size & 0.00\\

 & Line Distance Normalized by Method Size & 0.00\\

\multirow[t]{-11}{*}{\raggedright\arraybackslash Line (Old version):21, Line (New version):22} & Line Distance Normalized by Compilation Unit Size & 0.00\\
\hline
\end{tabular}
\end{table}

\normalsize

\newpage

\begin{landscape}

\subsection{Example: including a statement before} \label{example_including_statement}

\small

\normalsize

In this example, a new statement is included before the alert.

\small

\normalsize

\scriptsize

\begin{Shaded}
\begin{Highlighting}[]
\CommentTok{/*  1-                 */}\KeywordTok{package}\ImportTok{ pack_x;                                                /*  1-                 */package pack_x;}                                                
\CommentTok{/*  2-                 */}                                                               \CommentTok{/*  2-                 */}                                                               
\CommentTok{/*  3-                 */}\KeywordTok{import}\ImportTok{ importX.function;                                       /*  3-                 */import importX.function;}                                       
\CommentTok{/*  4-                 */}                                                               \CommentTok{/*  4-                 */}                                                               
\CommentTok{/*  5-                 */}\KeywordTok{class}\NormalTok{ ClassX }\KeywordTok{extends}\NormalTok{ ClassY }\KeywordTok{implements}\NormalTok{ InterfX \{               }\CommentTok{/*  5-                 */}\KeywordTok{class}\NormalTok{ ClassX }\KeywordTok{extends}\NormalTok{ ClassY }\KeywordTok{implements}\NormalTok{ InterfX \{               }
\CommentTok{/*  6-                 */}    \KeywordTok{private} \DataTypeTok{long}\NormalTok{ fieldX;                                       }\CommentTok{/*  6-                 */}    \KeywordTok{private} \DataTypeTok{long}\NormalTok{ fieldX;                                       }
\CommentTok{/*  7-                 */}                                                               \CommentTok{/*  7-                 */}                                                               
\CommentTok{/*  8-                 */}    \FunctionTok{ClassX}\NormalTok{(}\DataTypeTok{int}\NormalTok{ paramX, }\DataTypeTok{double}\NormalTok{ paramY) \{                                }\CommentTok{/*  8-                 */}    \FunctionTok{ClassX}\NormalTok{(}\DataTypeTok{int}\NormalTok{ paramX, }\DataTypeTok{double}\NormalTok{ paramY) \{                                }
\CommentTok{/*  9-                 */}        \DataTypeTok{int}\NormalTok{ varX = }\FunctionTok{function}\NormalTok{(paramX, paramY);                          }\CommentTok{/*  9-                 */}        \DataTypeTok{int}\NormalTok{ varX = }\FunctionTok{function}\NormalTok{(paramX, paramY);                           }
\CommentTok{/*   -                 *//*XXXXXXXXXXXXXXXXXXXXXXXXXXXXXXXXXXXXXX*/}                     \CommentTok{/* 10-                 */}        \KeywordTok{if}\NormalTok{ (varX == }\DecValTok{0}\NormalTok{)                                         }
\CommentTok{/* 10-                 */}        \KeywordTok{if}\NormalTok{ (varX == }\DecValTok{0}\NormalTok{)\{                                        }\CommentTok{/*   -                 *//*XXXXXXXXXXXXXXXXXXXXXXXXXXXXXXXXXXXXXX*/}                     
\CommentTok{/*   -                 *//*XXXXXXXXXXXXXXXXXXXXXXXXXXXXXXXXXXXXXX*/}                     \CommentTok{/* 11-                 */}\NormalTok{        \{                                                      }
\CommentTok{/* 11-                 */}            \KeywordTok{this}\NormalTok{.}\FunctionTok{fieldX}\NormalTok{ = }\DecValTok{1}\NormalTok{;                                   }\CommentTok{/* 12-                 */}            \KeywordTok{this}\NormalTok{.}\FunctionTok{fieldX}\NormalTok{ = }\DecValTok{1}\NormalTok{;                                   }
\CommentTok{/* 12-                 */}\NormalTok{        \}                                                      }\CommentTok{/*   -                 *//*XXXXXXXXXXXXXXXXXXXXXXXXXXXXXXXXXXXXXX*/}                     
\CommentTok{/*   -                 *//*XXXXXXXXXXXXXXXXXXXXXXXXXXXXXXXXXXXXXX*/}                     \CommentTok{/* 13-                 */}\NormalTok{        \}                                                                }
\CommentTok{/* 13-                 */}        \KeywordTok{else}\NormalTok{\{                                                  }\CommentTok{/* 14-                 */}        \KeywordTok{else}\NormalTok{\{                                                  }
\CommentTok{/* 14-                 */}            \KeywordTok{this}\NormalTok{.}\FunctionTok{fieldX}\NormalTok{ = }\DecValTok{0}\NormalTok{;                                   }\CommentTok{/* 15-                 */}            \KeywordTok{this}\NormalTok{.}\FunctionTok{fieldX}\NormalTok{ = }\DecValTok{0}\NormalTok{;                                   }
\CommentTok{/* 15-                 */}\NormalTok{        \}                                                      }\CommentTok{/* 16-                 */}\NormalTok{        \}                                                      }
\CommentTok{/* 16-                 */}\NormalTok{    \}                                                          }\CommentTok{/* 17-                 */}\NormalTok{    \}                                                          }
\CommentTok{/* 17-                 */}    \AttributeTok{@Override}                                                  \CommentTok{/* 18-                 */}    \AttributeTok{@Override}                                                  
\CommentTok{/* 18-                 */}    \KeywordTok{public} \DataTypeTok{int} \FunctionTok{methodX}\NormalTok{(}\DataTypeTok{int}\NormalTok{ paramW, }\BuiltInTok{Boolean}\NormalTok{ paramZ)             }\CommentTok{/* 19-                 */}    \KeywordTok{public} \DataTypeTok{int} \FunctionTok{methodX}\NormalTok{(}\DataTypeTok{int}\NormalTok{ paramW, }\BuiltInTok{Boolean}\NormalTok{ paramZ)             }
\CommentTok{/* 19-                 */}\NormalTok{    \{                                                          }\CommentTok{/* 20-                 */}\NormalTok{    \{                                                          }
\CommentTok{/*   -                 *//*XXXXXXXXXXXXXXXXXXXXXXXXXXXXXXXXXXXXXX*/}                     \CommentTok{/* 21-                 */}\NormalTok{        paramT = }\DecValTok{0}\NormalTok{;                                            }
\CommentTok{/* 20-                 */}        \KeywordTok{if}\NormalTok{ (paramZ)                                            }\CommentTok{/* 22-                 */}        \KeywordTok{if}\NormalTok{ (paramZ)                                            }
\CommentTok{/* 21-ControlStateme   */}\NormalTok{            fieldX = paramW;                                   }\CommentTok{/* 23-ControlStateme   */}\NormalTok{            fieldX = paramW;                                   }
\CommentTok{/* 22-                 */}        \KeywordTok{else}\NormalTok{\{                                                  }\CommentTok{/* 24-                 */}        \KeywordTok{else}\NormalTok{\{                                                  }
\CommentTok{/* 23-                 */}\NormalTok{            fieldX = }\DecValTok{0}\NormalTok{;                                        }\CommentTok{/* 25-                 */}\NormalTok{            fieldX = }\DecValTok{0}\NormalTok{;                                        }
\CommentTok{/* 24-                 */}\NormalTok{        \}                                                      }\CommentTok{/* 26-                 */}\NormalTok{        \}                                                      }
\CommentTok{/* 25-                 */}        \KeywordTok{return}\NormalTok{ paramW + }\KeywordTok{this}\NormalTok{.}\FunctionTok{fieldX}\NormalTok{;                           }\CommentTok{/* 27-                 */}        \KeywordTok{return}\NormalTok{ paramW + }\KeywordTok{this}\NormalTok{.}\FunctionTok{fieldX}\NormalTok{;                           }
\CommentTok{/* 26-                 */}\NormalTok{     \}                                                         }\CommentTok{/* 28-                 */}\NormalTok{     \}                                                         }
\CommentTok{/* 27-                 */}\NormalTok{\}                                                              }\CommentTok{/* 29-                 */}\NormalTok{\}                                                              }
\end{Highlighting}
\end{Shaded}

\normalsize

\begin{figure}
\centering
\includegraphics{figures/fake.png}
\caption{Comparison between old and new version
\label{comparison_include_statement_before}}
\end{figure}

\end{landscape}

\newpage

In the Table \ref{include_statement_before} we can see that the features
are not affected by this new statement.

\small

\begin{table}[!h]

\caption{\label{tab:unnamed-chunk-10}Resulting features: statement included before \label{include_statement_before} }
\centering
\begin{tabular}[t]{l|l|l}
\hline
Alert combination & Feature & Value\\
\hline
\rowcolor{gray!6}   & Same Rule & TRUE\\

 & Same Group ID & TRUE\\

\rowcolor{gray!6}   & Same Method Group ID & TRUE\\

 & Same Method Name & TRUE\\

\rowcolor{gray!6}   & Same Block & TRUE\\

 & Same Code & TRUE\\

\rowcolor{gray!6}   & Same Method Code & FALSE\\

 & Line Distance & 0.00\\

\rowcolor{gray!6}   & Line Distance Normalized by Block Size & 0.00\\

 & Line Distance Normalized by Method Size & 0.00\\

\multirow[t]{-11}{*}{\raggedright\arraybackslash Line (Old version):21, Line (New version):23} & Line Distance Normalized by Compilation Unit Size & 0.00\\
\hline
\end{tabular}
\end{table}

\normalsize

\newpage

\begin{landscape}

\subsection{Example: nesting the alert in an if statement} \label{example_nested_in_other_if}

\small

\normalsize

\scriptsize

\begin{Shaded}
\begin{Highlighting}[]
\CommentTok{/*  1-                 */}\KeywordTok{package}\ImportTok{ pack_x;                                                /*  1-                 */package pack_x;}                                                
\CommentTok{/*  2-                 */}                                                               \CommentTok{/*  2-                 */}                                                               
\CommentTok{/*  3-                 */}\KeywordTok{import}\ImportTok{ importX.function;                                       /*  3-                 */import importX.function;}                                       
\CommentTok{/*  4-                 */}                                                               \CommentTok{/*  4-                 */}                                                               
\CommentTok{/*  5-                 */}\KeywordTok{class}\NormalTok{ ClassX }\KeywordTok{extends}\NormalTok{ ClassY }\KeywordTok{implements}\NormalTok{ InterfX \{               }\CommentTok{/*  5-                 */}\KeywordTok{class}\NormalTok{ ClassX }\KeywordTok{extends}\NormalTok{ ClassY }\KeywordTok{implements}\NormalTok{ InterfX \{               }
\CommentTok{/*  6-                 */}    \KeywordTok{private} \DataTypeTok{long}\NormalTok{ fieldX;                                       }\CommentTok{/*  6-                 */}    \KeywordTok{private} \DataTypeTok{long}\NormalTok{ fieldX;                                       }
\CommentTok{/*  7-                 */}                                                               \CommentTok{/*  7-                 */}                                                               
\CommentTok{/*  8-                 */}    \FunctionTok{ClassX}\NormalTok{(}\DataTypeTok{int}\NormalTok{ paramX, }\DataTypeTok{double}\NormalTok{ paramY) \{                                }\CommentTok{/*  8-                 */}    \FunctionTok{ClassX}\NormalTok{(}\DataTypeTok{int}\NormalTok{ paramX, }\DataTypeTok{double}\NormalTok{ paramY) \{                                }
\CommentTok{/*  9-                 */}        \DataTypeTok{int}\NormalTok{ varX = }\FunctionTok{function}\NormalTok{(paramX, paramY);                          }\CommentTok{/*  9-                 */}        \DataTypeTok{int}\NormalTok{ varX = }\FunctionTok{function}\NormalTok{(paramX, paramY);                           }
\CommentTok{/*   -                 *//*XXXXXXXXXXXXXXXXXXXXXXXXXXXXXXXXXXXXXX*/}                     \CommentTok{/* 10-                 */}        \KeywordTok{if}\NormalTok{ (varX == }\DecValTok{0}\NormalTok{)                                         }
\CommentTok{/* 10-                 */}        \KeywordTok{if}\NormalTok{ (varX == }\DecValTok{0}\NormalTok{)\{                                        }\CommentTok{/*   -                 *//*XXXXXXXXXXXXXXXXXXXXXXXXXXXXXXXXXXXXXX*/}                     
\CommentTok{/*   -                 *//*XXXXXXXXXXXXXXXXXXXXXXXXXXXXXXXXXXXXXX*/}                     \CommentTok{/* 11-                 */}\NormalTok{        \{                                                      }
\CommentTok{/* 11-                 */}            \KeywordTok{this}\NormalTok{.}\FunctionTok{fieldX}\NormalTok{ = }\DecValTok{1}\NormalTok{;                                   }\CommentTok{/* 12-                 */}            \KeywordTok{this}\NormalTok{.}\FunctionTok{fieldX}\NormalTok{ = }\DecValTok{1}\NormalTok{;                                   }
\CommentTok{/* 12-                 */}\NormalTok{        \}                                                      }\CommentTok{/*   -                 *//*XXXXXXXXXXXXXXXXXXXXXXXXXXXXXXXXXXXXXX*/}                     
\CommentTok{/*   -                 *//*XXXXXXXXXXXXXXXXXXXXXXXXXXXXXXXXXXXXXX*/}                     \CommentTok{/* 13-                 */}\NormalTok{        \}                                                                }
\CommentTok{/* 13-                 */}        \KeywordTok{else}\NormalTok{\{                                                  }\CommentTok{/* 14-                 */}        \KeywordTok{else}\NormalTok{\{                                                  }
\CommentTok{/* 14-                 */}            \KeywordTok{this}\NormalTok{.}\FunctionTok{fieldX}\NormalTok{ = }\DecValTok{0}\NormalTok{;                                   }\CommentTok{/* 15-                 */}            \KeywordTok{this}\NormalTok{.}\FunctionTok{fieldX}\NormalTok{ = }\DecValTok{0}\NormalTok{;                                   }
\CommentTok{/* 15-                 */}\NormalTok{        \}                                                      }\CommentTok{/* 16-                 */}\NormalTok{        \}                                                      }
\CommentTok{/* 16-                 */}\NormalTok{    \}                                                          }\CommentTok{/* 17-                 */}\NormalTok{    \}                                                          }
\CommentTok{/* 17-                 */}    \AttributeTok{@Override}                                                  \CommentTok{/* 18-                 */}    \AttributeTok{@Override}                                                  
\CommentTok{/* 18-                 */}    \KeywordTok{public} \DataTypeTok{int} \FunctionTok{methodX}\NormalTok{(}\DataTypeTok{int}\NormalTok{ paramW, }\BuiltInTok{Boolean}\NormalTok{ paramZ)             }\CommentTok{/* 19-                 */}    \KeywordTok{public} \DataTypeTok{int} \FunctionTok{methodX}\NormalTok{(}\DataTypeTok{int}\NormalTok{ paramW, }\BuiltInTok{Boolean}\NormalTok{ paramZ)             }
\CommentTok{/* 19-                 */}\NormalTok{    \{                                                          }\CommentTok{/* 20-                 */}\NormalTok{    \{                                                          }
\CommentTok{/* 20-                 */}        \KeywordTok{if}\NormalTok{ (paramZ)                                            }\CommentTok{/*   -                 *//*XXXXXXXXXXXXXXXXXXXXXXXXXXXXXXXXXXXXXX*/}                     
\CommentTok{/*   -                 *//*XXXXXXXXXXXXXXXXXXXXXXXXXXXXXXXXXXXXXX*/}                     \CommentTok{/* 21-                 */}        \KeywordTok{if}\NormalTok{(paramZ == }\DecValTok{0}\NormalTok{)\{                                       }
\CommentTok{/* 21-ControlStateme   */}\NormalTok{            fieldX = paramW;                                   }\CommentTok{/*   -                 *//*XXXXXXXXXXXXXXXXXXXXXXXXXXXXXXXXXXXXXX*/}                     
\CommentTok{/*   -                 *//*XXXXXXXXXXXXXXXXXXXXXXXXXXXXXXXXXXXXXX*/}                     \CommentTok{/* 22-                 */}            \KeywordTok{if}\NormalTok{ (paramZ)                                        }
\CommentTok{/*   -                 *//*XXXXXXXXXXXXXXXXXXXXXXXXXXXXXXXXXXXXXX*/}                     \CommentTok{/* 23-ControlStateme   */}\NormalTok{                fieldX = paramW;                               }
\CommentTok{/* 23-                 */}\NormalTok{            fieldX = }\DecValTok{0}\NormalTok{;                                        }\CommentTok{/*   -                 *//*XXXXXXXXXXXXXXXXXXXXXXXXXXXXXXXXXXXXXX*/}                     
\CommentTok{/*   -                 *//*XXXXXXXXXXXXXXXXXXXXXXXXXXXXXXXXXXXXXX*/}                     \CommentTok{/* 24-                 */}            \KeywordTok{else}\NormalTok{\{                                              }
\CommentTok{/*   -                 *//*XXXXXXXXXXXXXXXXXXXXXXXXXXXXXXXXXXXXXX*/}                     \CommentTok{/* 25-                 */}\NormalTok{                fieldX = }\DecValTok{0}\NormalTok{;                                    }
\CommentTok{/*   -                 *//*XXXXXXXXXXXXXXXXXXXXXXXXXXXXXXXXXXXXXX*/}                     \CommentTok{/* 26-                 */}\NormalTok{            \}                                                  }
\CommentTok{/*   -                 *//*XXXXXXXXXXXXXXXXXXXXXXXXXXXXXXXXXXXXXX*/}                     \CommentTok{/* 27-                 */}\NormalTok{        \}                                                      }
\CommentTok{/* 22-                 */}        \KeywordTok{else}\NormalTok{\{                                                  }\CommentTok{/* 28-                 */}        \KeywordTok{else}\NormalTok{\{                                                  }
\CommentTok{/*   -                 *//*XXXXXXXXXXXXXXXXXXXXXXXXXXXXXXXXXXXXXX*/}                     \CommentTok{/* 29-                 */}\NormalTok{            fieldX = }\DecValTok{1}\NormalTok{;                                        }
\CommentTok{/* 24-                 */}\NormalTok{        \}                                                      }\CommentTok{/* 30-                 */}\NormalTok{        \}                                                      }
\CommentTok{/* 25-                 */}        \KeywordTok{return}\NormalTok{ paramW + }\KeywordTok{this}\NormalTok{.}\FunctionTok{fieldX}\NormalTok{;                           }\CommentTok{/* 31-                 */}        \KeywordTok{return}\NormalTok{ paramW + }\KeywordTok{this}\NormalTok{.}\FunctionTok{fieldX}\NormalTok{;                           }
\CommentTok{/* 26-                 */}\NormalTok{     \}                                                         }\CommentTok{/* 32-                 */}\NormalTok{     \}                                                         }
\CommentTok{/* 27-                 */}\NormalTok{\}                                                              }\CommentTok{/* 33-                 */}\NormalTok{\}                                                              }
\end{Highlighting}
\end{Shaded}

\normalsize

\begin{figure}
\centering
\includegraphics{figures/fake.png}
\caption{Comparison between old and new version
\label{comparison_nested_in_other_if}}
\end{figure}

\end{landscape}

\newpage

In the Table \ref{nested_in_other_if} we can see that the algorithm does
not recognize the two nodes as equivalent, but other features can lead
us to the conclusion that the alert is still open.

\small

\begin{table}[!h]

\caption{\label{tab:unnamed-chunk-12}Resulting features: statement included before \label{nested_in_other_if} }
\centering
\begin{tabular}[t]{l|l|l}
\hline
Alert combination & Feature & Value\\
\hline
\rowcolor{gray!6}   & Same Rule & TRUE\\

 & Same Group ID & FALSE\\

\rowcolor{gray!6}   & Same Method Group ID & TRUE\\

 & Same Method Name & TRUE\\

\rowcolor{gray!6}   & Same Block & FALSE\\

 & Same Code & TRUE\\

\rowcolor{gray!6}   & Same Method Code & FALSE\\

 & Line Distance & 2.00\\

\rowcolor{gray!6}   & Line Distance Normalized by Block Size & 0.13\\

 & Line Distance Normalized by Method Size & 0.12\\

\multirow[t]{-11}{*}{\raggedright\arraybackslash Line (Old version):21, Line (New version):23} & Line Distance Normalized by Compilation Unit Size & 0.05\\
\hline
\end{tabular}
\end{table}

\normalsize

\newpage

\begin{landscape}

\subsection{Example: editing the line that generates the alert} \label{example_editing_line}

\small

\normalsize

\scriptsize

\begin{Shaded}
\begin{Highlighting}[]
\CommentTok{/*  1-                 */}\KeywordTok{package}\ImportTok{ pack_x;                                                /*  1-                 */package pack_x;}                                                
\CommentTok{/*  2-                 */}                                                               \CommentTok{/*  2-                 */}                                                               
\CommentTok{/*  3-                 */}\KeywordTok{import}\ImportTok{ importX.function;                                       /*  3-                 */import importX.function;}                                       
\CommentTok{/*  4-                 */}                                                               \CommentTok{/*  4-                 */}                                                               
\CommentTok{/*  5-                 */}\KeywordTok{class}\NormalTok{ ClassX }\KeywordTok{extends}\NormalTok{ ClassY }\KeywordTok{implements}\NormalTok{ InterfX \{               }\CommentTok{/*  5-                 */}\KeywordTok{class}\NormalTok{ ClassX }\KeywordTok{extends}\NormalTok{ ClassY }\KeywordTok{implements}\NormalTok{ InterfX \{               }
\CommentTok{/*  6-                 */}    \KeywordTok{private} \DataTypeTok{long}\NormalTok{ fieldX;                                       }\CommentTok{/*  6-                 */}    \KeywordTok{private} \DataTypeTok{long}\NormalTok{ fieldX;                                       }
\CommentTok{/*  7-                 */}                                                               \CommentTok{/*  7-                 */}                                                               
\CommentTok{/*  8-                 */}    \FunctionTok{ClassX}\NormalTok{(}\DataTypeTok{int}\NormalTok{ paramX, }\DataTypeTok{double}\NormalTok{ paramY) \{                                }\CommentTok{/*  8-                 */}    \FunctionTok{ClassX}\NormalTok{(}\DataTypeTok{int}\NormalTok{ paramX, }\DataTypeTok{double}\NormalTok{ paramY) \{                                }
\CommentTok{/*  9-                 */}        \DataTypeTok{int}\NormalTok{ varX = }\FunctionTok{function}\NormalTok{(paramX, paramY);                          }\CommentTok{/*  9-                 */}        \DataTypeTok{int}\NormalTok{ varX = }\FunctionTok{function}\NormalTok{(paramX, paramY);                           }
\CommentTok{/*   -                 *//*XXXXXXXXXXXXXXXXXXXXXXXXXXXXXXXXXXXXXX*/}                     \CommentTok{/* 10-                 */}        \KeywordTok{if}\NormalTok{ (varX == }\DecValTok{0}\NormalTok{)                                         }
\CommentTok{/* 10-                 */}        \KeywordTok{if}\NormalTok{ (varX == }\DecValTok{0}\NormalTok{)\{                                        }\CommentTok{/*   -                 *//*XXXXXXXXXXXXXXXXXXXXXXXXXXXXXXXXXXXXXX*/}                     
\CommentTok{/*   -                 *//*XXXXXXXXXXXXXXXXXXXXXXXXXXXXXXXXXXXXXX*/}                     \CommentTok{/* 11-                 */}\NormalTok{        \{                                                      }
\CommentTok{/* 11-                 */}            \KeywordTok{this}\NormalTok{.}\FunctionTok{fieldX}\NormalTok{ = }\DecValTok{1}\NormalTok{;                                   }\CommentTok{/* 12-                 */}            \KeywordTok{this}\NormalTok{.}\FunctionTok{fieldX}\NormalTok{ = }\DecValTok{1}\NormalTok{;                                   }
\CommentTok{/* 12-                 */}\NormalTok{        \}                                                      }\CommentTok{/*   -                 *//*XXXXXXXXXXXXXXXXXXXXXXXXXXXXXXXXXXXXXX*/}                     
\CommentTok{/*   -                 *//*XXXXXXXXXXXXXXXXXXXXXXXXXXXXXXXXXXXXXX*/}                     \CommentTok{/* 13-                 */}\NormalTok{        \}                                                                }
\CommentTok{/* 13-                 */}        \KeywordTok{else}\NormalTok{\{                                                  }\CommentTok{/* 14-                 */}        \KeywordTok{else}\NormalTok{\{                                                  }
\CommentTok{/* 14-                 */}            \KeywordTok{this}\NormalTok{.}\FunctionTok{fieldX}\NormalTok{ = }\DecValTok{0}\NormalTok{;                                   }\CommentTok{/* 15-                 */}            \KeywordTok{this}\NormalTok{.}\FunctionTok{fieldX}\NormalTok{ = }\DecValTok{0}\NormalTok{;                                   }
\CommentTok{/* 15-                 */}\NormalTok{        \}                                                      }\CommentTok{/* 16-                 */}\NormalTok{        \}                                                      }
\CommentTok{/* 16-                 */}\NormalTok{    \}                                                          }\CommentTok{/* 17-                 */}\NormalTok{    \}                                                          }
\CommentTok{/* 17-                 */}    \AttributeTok{@Override}                                                  \CommentTok{/* 18-                 */}    \AttributeTok{@Override}                                                  
\CommentTok{/* 18-                 */}    \KeywordTok{public} \DataTypeTok{int} \FunctionTok{methodX}\NormalTok{(}\DataTypeTok{int}\NormalTok{ paramW, }\BuiltInTok{Boolean}\NormalTok{ paramZ)             }\CommentTok{/* 19-                 */}    \KeywordTok{public} \DataTypeTok{int} \FunctionTok{methodX}\NormalTok{(}\DataTypeTok{int}\NormalTok{ paramW, }\BuiltInTok{Boolean}\NormalTok{ paramZ)             }
\CommentTok{/* 19-                 */}\NormalTok{    \{                                                          }\CommentTok{/* 20-                 */}\NormalTok{    \{                                                          }
\CommentTok{/* 20-                 */}        \KeywordTok{if}\NormalTok{ (paramZ)                                            }\CommentTok{/* 21-                 */}        \KeywordTok{if}\NormalTok{ (paramZ)                                            }
\CommentTok{/* 21-ControlStateme   */}\NormalTok{            fieldX = paramW;                                   }\CommentTok{/*   -                 *//*XXXXXXXXXXXXXXXXXXXXXXXXXXXXXXXXXXXXXX*/}                     
\CommentTok{/*   -                 *//*XXXXXXXXXXXXXXXXXXXXXXXXXXXXXXXXXXXXXX*/}                     \CommentTok{/* 22-ControlStateme   */}\NormalTok{            fieldX = paramZ;                                   }
\CommentTok{/* 22-                 */}        \KeywordTok{else}\NormalTok{\{                                                  }\CommentTok{/* 23-                 */}        \KeywordTok{else}\NormalTok{\{                                                  }
\CommentTok{/* 23-                 */}\NormalTok{            fieldX = }\DecValTok{0}\NormalTok{;                                        }\CommentTok{/* 24-                 */}\NormalTok{            fieldX = }\DecValTok{0}\NormalTok{;                                        }
\CommentTok{/* 24-                 */}\NormalTok{        \}                                                      }\CommentTok{/* 25-                 */}\NormalTok{        \}                                                      }
\CommentTok{/* 25-                 */}        \KeywordTok{return}\NormalTok{ paramW + }\KeywordTok{this}\NormalTok{.}\FunctionTok{fieldX}\NormalTok{;                           }\CommentTok{/* 26-                 */}        \KeywordTok{return}\NormalTok{ paramW + }\KeywordTok{this}\NormalTok{.}\FunctionTok{fieldX}\NormalTok{;                           }
\CommentTok{/* 26-                 */}\NormalTok{     \}                                                         }\CommentTok{/* 27-                 */}\NormalTok{     \}                                                         }
\CommentTok{/* 27-                 */}\NormalTok{\}                                                              }\CommentTok{/* 28-                 */}\NormalTok{\}                                                              }
\end{Highlighting}
\end{Shaded}

\normalsize

\begin{figure}
\centering
\includegraphics{figures/fake.png}
\caption{Comparison between old and new version
\label{comparison_editing_line}}
\end{figure}

\end{landscape}

\newpage

In the Table \ref{editing_line} the nodes are not recognized as
equivalent, but other features can lead us to the conclusion that the
alert is still open.

\small

\begin{table}[!h]

\caption{\label{tab:unnamed-chunk-14}Resulting features: alert line edited \label{editing_line} }
\centering
\begin{tabular}[t]{l|l|l}
\hline
Alert combination & Feature & Value\\
\hline
\rowcolor{gray!6}   & Same Rule & TRUE\\

 & Same Group ID & FALSE\\

\rowcolor{gray!6}   & Same Method Group ID & TRUE\\

 & Same Method Name & TRUE\\

\rowcolor{gray!6}   & Same Block & TRUE\\

 & Same Code & FALSE\\

\rowcolor{gray!6}   & Same Method Code & FALSE\\

 & Line Distance & 1.00\\

\rowcolor{gray!6}   & Line Distance Normalized by Block Size & 0.20\\

 & Line Distance Normalized by Method Size & 0.11\\

\multirow[t]{-11}{*}{\raggedright\arraybackslash Line (Old version):21, Line (New version):22} & Line Distance Normalized by Compilation Unit Size & 0.03\\
\hline
\end{tabular}
\end{table}

\normalsize

\newpage

\begin{landscape}

\subsection{Example: changing the order of the methods} \label{example_editing_line}

We changed the order of the methods in two ways

\small

\normalsize

\scriptsize

\begin{Shaded}
\begin{Highlighting}[]
\CommentTok{/*  1-                 */}\KeywordTok{package}\ImportTok{ pack_x;                                                /*  1-                 */package pack_x;}                                                
\CommentTok{/*  2-                 */}                                                               \CommentTok{/*  2-                 */}                                                               
\CommentTok{/*  3-                 */}\KeywordTok{import}\ImportTok{ importX.function;                                       /*  3-                 */import importX.function;}                                       
\CommentTok{/*  4-                 */}                                                               \CommentTok{/*  4-                 */}                                                               
\CommentTok{/*  5-                 */}\KeywordTok{class}\NormalTok{ ClassX }\KeywordTok{extends}\NormalTok{ ClassY }\KeywordTok{implements}\NormalTok{ InterfX \{               }\CommentTok{/*  5-                 */}\KeywordTok{class}\NormalTok{ ClassX }\KeywordTok{extends}\NormalTok{ ClassY }\KeywordTok{implements}\NormalTok{ InterfX \{               }
\CommentTok{/*  6-                 */}    \KeywordTok{private} \DataTypeTok{long}\NormalTok{ fieldX;                                       }\CommentTok{/*  6-                 */}    \KeywordTok{private} \DataTypeTok{long}\NormalTok{ fieldX;                                       }
\CommentTok{/*   -                 *//*XXXXXXXXXXXXXXXXXXXXXXXXXXXXXXXXXXXXXX*/}                     \CommentTok{/*  7-                 */}                                                               
\CommentTok{/*  7-                 */}                                                               \CommentTok{/*   -                 *//*XXXXXXXXXXXXXXXXXXXXXXXXXXXXXXXXXXXXXX*/}                     
\CommentTok{/*  8-                 */}    \FunctionTok{ClassX}\NormalTok{(}\DataTypeTok{int}\NormalTok{ paramX, }\DataTypeTok{double}\NormalTok{ paramY) \{                                }\CommentTok{/*   -                 *//*XXXXXXXXXXXXXXXXXXXXXXXXXXXXXXXXXXXXXX*/}                     
\CommentTok{/*  9-                 */}        \DataTypeTok{int}\NormalTok{ varX = }\FunctionTok{function}\NormalTok{(paramX, paramY);                          }\CommentTok{/*   -                 *//*XXXXXXXXXXXXXXXXXXXXXXXXXXXXXXXXXXXXXX*/}                     
\CommentTok{/* 10-                 */}        \KeywordTok{if}\NormalTok{ (varX == }\DecValTok{0}\NormalTok{)\{                                        }\CommentTok{/*   -                 *//*XXXXXXXXXXXXXXXXXXXXXXXXXXXXXXXXXXXXXX*/}                     
\CommentTok{/* 11-                 */}            \KeywordTok{this}\NormalTok{.}\FunctionTok{fieldX}\NormalTok{ = }\DecValTok{1}\NormalTok{;                                   }\CommentTok{/*   -                 *//*XXXXXXXXXXXXXXXXXXXXXXXXXXXXXXXXXXXXXX*/}                     
\CommentTok{/* 12-                 */}\NormalTok{        \}                                                      }\CommentTok{/*   -                 *//*XXXXXXXXXXXXXXXXXXXXXXXXXXXXXXXXXXXXXX*/}                     
\CommentTok{/* 13-                 */}        \KeywordTok{else}\NormalTok{\{                                                  }\CommentTok{/*   -                 *//*XXXXXXXXXXXXXXXXXXXXXXXXXXXXXXXXXXXXXX*/}                     
\CommentTok{/* 14-                 */}            \KeywordTok{this}\NormalTok{.}\FunctionTok{fieldX}\NormalTok{ = }\DecValTok{0}\NormalTok{;                                   }\CommentTok{/*   -                 *//*XXXXXXXXXXXXXXXXXXXXXXXXXXXXXXXXXXXXXX*/}                     
\CommentTok{/* 15-                 */}\NormalTok{        \}                                                      }\CommentTok{/*   -                 *//*XXXXXXXXXXXXXXXXXXXXXXXXXXXXXXXXXXXXXX*/}                     
\CommentTok{/* 16-                 */}\NormalTok{    \}                                                          }\CommentTok{/*   -                 *//*XXXXXXXXXXXXXXXXXXXXXXXXXXXXXXXXXXXXXX*/}                     
\CommentTok{/* 17-                 */}    \AttributeTok{@Override}                                                  \CommentTok{/*  8-                 */}    \AttributeTok{@Override}                                                  
\CommentTok{/* 18-                 */}    \KeywordTok{public} \DataTypeTok{int} \FunctionTok{methodX}\NormalTok{(}\DataTypeTok{int}\NormalTok{ paramW, }\BuiltInTok{Boolean}\NormalTok{ paramZ)             }\CommentTok{/*  9-                 */}    \KeywordTok{public} \DataTypeTok{int} \FunctionTok{methodX}\NormalTok{(}\DataTypeTok{int}\NormalTok{ paramW, }\BuiltInTok{Boolean}\NormalTok{ paramZ)             }
\CommentTok{/*   -                 *//*XXXXXXXXXXXXXXXXXXXXXXXXXXXXXXXXXXXXXX*/}                     \CommentTok{/* 18-                 */}                                                               
\CommentTok{/* 19-                 */}\NormalTok{    \{                                                          }\CommentTok{/* 10-                 */}\NormalTok{    \{                                                          }
\CommentTok{/*   -                 *//*XXXXXXXXXXXXXXXXXXXXXXXXXXXXXXXXXXXXXX*/}                     \CommentTok{/* 19-                 */}                                                               
\CommentTok{/* 20-                 */}        \KeywordTok{if}\NormalTok{ (paramZ)                                            }\CommentTok{/* 11-                 */}        \KeywordTok{if}\NormalTok{ (paramZ)                                            }
\CommentTok{/*   -                 *//*XXXXXXXXXXXXXXXXXXXXXXXXXXXXXXXXXXXXXX*/}                     \CommentTok{/* 20-                 */}    \FunctionTok{ClassX}\NormalTok{(}\DataTypeTok{int}\NormalTok{ paramX, }\DataTypeTok{double}\NormalTok{ paramY) \{                                }
\CommentTok{/* 21-ControlStateme   */}\NormalTok{            fieldX = paramW;                                   }\CommentTok{/* 12-ControlStateme   */}\NormalTok{            fieldX = paramW;                                   }
\CommentTok{/*   -                 *//*XXXXXXXXXXXXXXXXXXXXXXXXXXXXXXXXXXXXXX*/}                     \CommentTok{/* 21-                 */}        \DataTypeTok{int}\NormalTok{ varX = }\FunctionTok{function}\NormalTok{(paramX, paramY);                          }
\CommentTok{/* 22-                 */}        \KeywordTok{else}\NormalTok{\{                                                  }\CommentTok{/* 13-                 */}        \KeywordTok{else}\NormalTok{\{                                                  }
\CommentTok{/*   -                 *//*XXXXXXXXXXXXXXXXXXXXXXXXXXXXXXXXXXXXXX*/}                     \CommentTok{/* 22-                 */}        \KeywordTok{if}\NormalTok{ (varX == }\DecValTok{0}\NormalTok{)                                         }
\CommentTok{/* 23-                 */}\NormalTok{            fieldX = }\DecValTok{0}\NormalTok{;                                        }\CommentTok{/* 14-                 */}\NormalTok{            fieldX = }\DecValTok{0}\NormalTok{;                                        }
\CommentTok{/*   -                 *//*XXXXXXXXXXXXXXXXXXXXXXXXXXXXXXXXXXXXXX*/}                     \CommentTok{/* 23-                 */}\NormalTok{        \{                                                      }
\CommentTok{/* 24-                 */}\NormalTok{        \}                                                      }\CommentTok{/* 15-                 */}\NormalTok{        \}                                                      }
\CommentTok{/*   -                 *//*XXXXXXXXXXXXXXXXXXXXXXXXXXXXXXXXXXXXXX*/}                     \CommentTok{/* 24-                 */}            \KeywordTok{this}\NormalTok{.}\FunctionTok{fieldX}\NormalTok{ = }\DecValTok{1}\NormalTok{;                                   }
\CommentTok{/* 25-                 */}        \KeywordTok{return}\NormalTok{ paramW + }\KeywordTok{this}\NormalTok{.}\FunctionTok{fieldX}\NormalTok{;                           }\CommentTok{/* 16-                 */}        \KeywordTok{return}\NormalTok{ paramW + }\KeywordTok{this}\NormalTok{.}\FunctionTok{fieldX}\NormalTok{;                           }
\CommentTok{/*   -                 *//*XXXXXXXXXXXXXXXXXXXXXXXXXXXXXXXXXXXXXX*/}                     \CommentTok{/* 25-                 */}\NormalTok{        \}                                                                }
\CommentTok{/* 26-                 */}\NormalTok{     \}                                                         }\CommentTok{/* 17-                 */}\NormalTok{     \}                                                         }
\CommentTok{/*   -                 *//*XXXXXXXXXXXXXXXXXXXXXXXXXXXXXXXXXXXXXX*/}                     \CommentTok{/* 26-                 */}        \KeywordTok{else}\NormalTok{\{                                                  }
\CommentTok{/*   -                 *//*XXXXXXXXXXXXXXXXXXXXXXXXXXXXXXXXXXXXXX*/}                     \CommentTok{/* 27-                 */}            \KeywordTok{this}\NormalTok{.}\FunctionTok{fieldX}\NormalTok{ = }\DecValTok{0}\NormalTok{;                                   }
\CommentTok{/*   -                 *//*XXXXXXXXXXXXXXXXXXXXXXXXXXXXXXXXXXXXXX*/}                     \CommentTok{/* 28-                 */}\NormalTok{        \}                                                      }
\CommentTok{/*   -                 *//*XXXXXXXXXXXXXXXXXXXXXXXXXXXXXXXXXXXXXX*/}                     \CommentTok{/* 29-                 */}\NormalTok{    \}                                                          }
\CommentTok{/* 27-                 */}\NormalTok{\}                                                              }\CommentTok{/* 30-                 */}\NormalTok{\}                                                              }
\end{Highlighting}
\end{Shaded}

\normalsize

\begin{figure}
\centering
\includegraphics{figures/fake.png}
\caption{Comparison between old and new version
\label{comparison_changing_method_order}}
\end{figure}

\end{landscape}

\newpage

In the Table \ref{changing_method_order} the nodes are not recognized as
equivalent, but other features can lead us to the conclusion that the
alert is still open.

\small

\begin{table}[!h]

\caption{\label{tab:unnamed-chunk-16}Resulting features: changed methods order \label{changing_method_order} }
\centering
\begin{tabular}[t]{l|l|l}
\hline
Alert combination & Feature & Value\\
\hline
\rowcolor{gray!6}   & Same Rule & TRUE\\

 & Same Group ID & TRUE\\

\rowcolor{gray!6}   & Same Method Group ID & TRUE\\

 & Same Method Name & TRUE\\

\rowcolor{gray!6}   & Same Block & TRUE\\

 & Same Code & TRUE\\

\rowcolor{gray!6}   & Same Method Code & TRUE\\

 & Line Distance & 0.00\\

\rowcolor{gray!6}   & Line Distance Normalized by Block Size & 0.00\\

 & Line Distance Normalized by Method Size & 0.00\\

\multirow[t]{-11}{*}{\raggedright\arraybackslash Line (Old version):21, Line (New version):12} & Line Distance Normalized by Compilation Unit Size & 0.00\\
\hline
\end{tabular}
\end{table}

\normalsize

\newpage

\begin{landscape}

\small

\normalsize

\scriptsize

\begin{Shaded}
\begin{Highlighting}[]
\CommentTok{/*  1-                 */}\KeywordTok{package}\ImportTok{ pack_x;                                                /*  1-                 */package pack_x;}                                                
\CommentTok{/*  2-                 */}                                                               \CommentTok{/*  2-                 */}                                                               
\CommentTok{/*  3-                 */}\KeywordTok{import}\ImportTok{ importX.function;                                       /*  3-                 */import importX.function;}                                       
\CommentTok{/*  4-                 */}                                                               \CommentTok{/*  4-                 */}                                                               
\CommentTok{/*  5-                 */}\KeywordTok{class}\NormalTok{ ClassX }\KeywordTok{extends}\NormalTok{ ClassY }\KeywordTok{implements}\NormalTok{ InterfX \{               }\CommentTok{/*  5-                 */}\KeywordTok{class}\NormalTok{ ClassX }\KeywordTok{extends}\NormalTok{ ClassY }\KeywordTok{implements}\NormalTok{ InterfX \{               }
\CommentTok{/*  6-                 */}    \KeywordTok{private} \DataTypeTok{long}\NormalTok{ fieldX;                                       }\CommentTok{/*  6-                 */}    \KeywordTok{private} \DataTypeTok{long}\NormalTok{ fieldX;                                       }
\CommentTok{/*   -                 *//*XXXXXXXXXXXXXXXXXXXXXXXXXXXXXXXXXXXXXX*/}                     \CommentTok{/*  7-                 */}                                                               
\CommentTok{/*  7-                 */}                                                               \CommentTok{/*   -                 *//*XXXXXXXXXXXXXXXXXXXXXXXXXXXXXXXXXXXXXX*/}                     
\CommentTok{/*   -                 *//*XXXXXXXXXXXXXXXXXXXXXXXXXXXXXXXXXXXXXX*/}                     \CommentTok{/*  8-                 */}    \FunctionTok{ClassX}\NormalTok{(}\DataTypeTok{int}\NormalTok{ paramX, }\DataTypeTok{double}\NormalTok{ paramY) \{                                }
\CommentTok{/*   -                 *//*XXXXXXXXXXXXXXXXXXXXXXXXXXXXXXXXXXXXXX*/}                     \CommentTok{/*  9-                 */}        \DataTypeTok{int}\NormalTok{ varX = }\FunctionTok{function}\NormalTok{(paramX, paramY);                          }
\CommentTok{/*   -                 *//*XXXXXXXXXXXXXXXXXXXXXXXXXXXXXXXXXXXXXX*/}                     \CommentTok{/* 10-                 */}        \KeywordTok{if}\NormalTok{ (varX == }\DecValTok{0}\NormalTok{)\{                                        }
\CommentTok{/*   -                 *//*XXXXXXXXXXXXXXXXXXXXXXXXXXXXXXXXXXXXXX*/}                     \CommentTok{/* 11-                 */}            \KeywordTok{this}\NormalTok{.}\FunctionTok{fieldX}\NormalTok{ = }\DecValTok{1}\NormalTok{;                                   }
\CommentTok{/*   -                 *//*XXXXXXXXXXXXXXXXXXXXXXXXXXXXXXXXXXXXXX*/}                     \CommentTok{/* 12-                 */}\NormalTok{        \}                                                      }
\CommentTok{/*   -                 *//*XXXXXXXXXXXXXXXXXXXXXXXXXXXXXXXXXXXXXX*/}                     \CommentTok{/* 13-                 */}        \KeywordTok{else}\NormalTok{\{                                                  }
\CommentTok{/*   -                 *//*XXXXXXXXXXXXXXXXXXXXXXXXXXXXXXXXXXXXXX*/}                     \CommentTok{/* 14-                 */}            \KeywordTok{this}\NormalTok{.}\FunctionTok{fieldX}\NormalTok{ = }\DecValTok{0}\NormalTok{;                                   }
\CommentTok{/*   -                 *//*XXXXXXXXXXXXXXXXXXXXXXXXXXXXXXXXXXXXXX*/}                     \CommentTok{/* 15-                 */}\NormalTok{        \}                                                      }
\CommentTok{/*   -                 *//*XXXXXXXXXXXXXXXXXXXXXXXXXXXXXXXXXXXXXX*/}                     \CommentTok{/* 16-                 */}\NormalTok{    \}                                                          }
\CommentTok{/*  8-                 */}    \AttributeTok{@Override}                                                  \CommentTok{/* 17-                 */}    \AttributeTok{@Override}                                                  
\CommentTok{/*  9-                 */}    \KeywordTok{public} \DataTypeTok{int} \FunctionTok{methodX}\NormalTok{(}\DataTypeTok{int}\NormalTok{ paramW, }\BuiltInTok{Boolean}\NormalTok{ paramZ)             }\CommentTok{/* 18-                 */}    \KeywordTok{public} \DataTypeTok{int} \FunctionTok{methodX}\NormalTok{(}\DataTypeTok{int}\NormalTok{ paramW, }\BuiltInTok{Boolean}\NormalTok{ paramZ)             }
\CommentTok{/* 18-                 */}                                                               \CommentTok{/*   -                 *//*XXXXXXXXXXXXXXXXXXXXXXXXXXXXXXXXXXXXXX*/}                     
\CommentTok{/* 10-                 */}\NormalTok{    \{                                                          }\CommentTok{/* 19-                 */}\NormalTok{    \{                                                          }
\CommentTok{/* 19-                 */}                                                               \CommentTok{/*   -                 *//*XXXXXXXXXXXXXXXXXXXXXXXXXXXXXXXXXXXXXX*/}                     
\CommentTok{/* 11-                 */}        \KeywordTok{if}\NormalTok{ (paramZ)                                            }\CommentTok{/* 20-                 */}        \KeywordTok{if}\NormalTok{ (paramZ)                                            }
\CommentTok{/* 20-                 */}    \FunctionTok{ClassX}\NormalTok{(}\DataTypeTok{int}\NormalTok{ paramX, }\DataTypeTok{double}\NormalTok{ paramY) \{                                }\CommentTok{/*   -                 *//*XXXXXXXXXXXXXXXXXXXXXXXXXXXXXXXXXXXXXX*/}                     
\CommentTok{/* 12-ControlStateme   */}\NormalTok{            fieldX = paramW;                                   }\CommentTok{/* 21-ControlStateme   */}\NormalTok{            fieldX = paramW;                                   }
\CommentTok{/* 21-                 */}        \DataTypeTok{int}\NormalTok{ varX = }\FunctionTok{function}\NormalTok{(paramX, paramY);                          }\CommentTok{/*   -                 *//*XXXXXXXXXXXXXXXXXXXXXXXXXXXXXXXXXXXXXX*/}                     
\CommentTok{/* 13-                 */}        \KeywordTok{else}\NormalTok{\{                                                  }\CommentTok{/* 22-                 */}        \KeywordTok{else}\NormalTok{\{                                                  }
\CommentTok{/* 22-                 */}        \KeywordTok{if}\NormalTok{ (varX == }\DecValTok{0}\NormalTok{)                                         }\CommentTok{/*   -                 *//*XXXXXXXXXXXXXXXXXXXXXXXXXXXXXXXXXXXXXX*/}                     
\CommentTok{/* 14-                 */}\NormalTok{            fieldX = }\DecValTok{0}\NormalTok{;                                        }\CommentTok{/* 23-                 */}\NormalTok{            fieldX = }\DecValTok{0}\NormalTok{;                                        }
\CommentTok{/* 23-                 */}\NormalTok{        \{                                                      }\CommentTok{/*   -                 *//*XXXXXXXXXXXXXXXXXXXXXXXXXXXXXXXXXXXXXX*/}                     
\CommentTok{/* 15-                 */}\NormalTok{        \}                                                      }\CommentTok{/* 24-                 */}\NormalTok{        \}                                                      }
\CommentTok{/* 24-                 */}            \KeywordTok{this}\NormalTok{.}\FunctionTok{fieldX}\NormalTok{ = }\DecValTok{1}\NormalTok{;                                   }\CommentTok{/*   -                 *//*XXXXXXXXXXXXXXXXXXXXXXXXXXXXXXXXXXXXXX*/}                     
\CommentTok{/* 16-                 */}        \KeywordTok{return}\NormalTok{ paramW + }\KeywordTok{this}\NormalTok{.}\FunctionTok{fieldX}\NormalTok{;                           }\CommentTok{/* 25-                 */}        \KeywordTok{return}\NormalTok{ paramW + }\KeywordTok{this}\NormalTok{.}\FunctionTok{fieldX}\NormalTok{;                           }
\CommentTok{/* 25-                 */}\NormalTok{        \}                                                                }\CommentTok{/*   -                 *//*XXXXXXXXXXXXXXXXXXXXXXXXXXXXXXXXXXXXXX*/}                     
\CommentTok{/* 17-                 */}\NormalTok{     \}                                                         }\CommentTok{/* 26-                 */}\NormalTok{     \}                                                         }
\CommentTok{/* 26-                 */}        \KeywordTok{else}\NormalTok{\{                                                  }\CommentTok{/*   -                 *//*XXXXXXXXXXXXXXXXXXXXXXXXXXXXXXXXXXXXXX*/}                     
\CommentTok{/* 27-                 */}            \KeywordTok{this}\NormalTok{.}\FunctionTok{fieldX}\NormalTok{ = }\DecValTok{0}\NormalTok{;                                   }\CommentTok{/*   -                 *//*XXXXXXXXXXXXXXXXXXXXXXXXXXXXXXXXXXXXXX*/}                     
\CommentTok{/* 28-                 */}\NormalTok{        \}                                                      }\CommentTok{/*   -                 *//*XXXXXXXXXXXXXXXXXXXXXXXXXXXXXXXXXXXXXX*/}                     
\CommentTok{/* 29-                 */}\NormalTok{    \}                                                          }\CommentTok{/*   -                 *//*XXXXXXXXXXXXXXXXXXXXXXXXXXXXXXXXXXXXXX*/}                     
\CommentTok{/* 30-                 */}\NormalTok{\}                                                              }\CommentTok{/* 27-                 */}\NormalTok{\}                                                              }
\CommentTok{/* 31-                 */}                                                               \CommentTok{/* 27-                 */}\NormalTok{\}                                                              }
\CommentTok{/* 32-                 *//*XXXXXXXXXXXXXXXXXXXXXXXXXXXXXXXXXXXXXX*/}                     \CommentTok{/* 28-                 *//*XXXXXXXXXXXXXXXXXXXXXXXXXXXXXXXXXXXXXX*/}                     
\end{Highlighting}
\end{Shaded}

\normalsize

\begin{figure}
\centering
\includegraphics{figures/fake.png}
\caption{Comparison between old and new version
\label{comparison_changing_method_order_2}}
\end{figure}

\end{landscape}

\newpage

In the Table \ref{changing_method_order_2} the nodes are not recognized
as equivalent, but other features can lead us to the conclusion that the
alert is still open.

\small

\begin{table}[!h]

\caption{\label{tab:unnamed-chunk-18}Resulting features: changed methods order 2 \label{changing_method_order_2} }
\centering
\begin{tabular}[t]{l|l|l}
\hline
Alert combination & Feature & Value\\
\hline
\rowcolor{gray!6}   & Same Rule & TRUE\\

 & Same Group ID & TRUE\\

\rowcolor{gray!6}   & Same Method Group ID & TRUE\\

 & Same Method Name & TRUE\\

\rowcolor{gray!6}   & Same Block & TRUE\\

 & Same Code & TRUE\\

\rowcolor{gray!6}   & Same Method Code & TRUE\\

 & Line Distance & 0.00\\

\rowcolor{gray!6}   & Line Distance Normalized by Block Size & 0.00\\

 & Line Distance Normalized by Method Size & 0.00\\

\multirow[t]{-11}{*}{\raggedright\arraybackslash Line (Old version):12, Line (New version):21} & Line Distance Normalized by Compilation Unit Size & 0.00\\
\hline
\end{tabular}
\end{table}

\normalsize

\subsection{Heuristic to decide based on the features}\label{heuristic}

The heuristic chosen in this work follow these rules when :

\begin{itemize}
\item If all the boolean features are TRUE and \textbf{Line Distance} is true, then the new and old alerts are declared the same and \textbf{Open};
\item If \textbf{Same Method Code} is TRUE, then we consider it's the same method, even if the name or the group id is not the same. But if the method code is the same, then the code of the node that generated the alert must be the same in the versions and the kind of the alert must be the same. So if the features \textbf{Same Rule}, \textbf{Same Method Code} and 
\textbf{Same Code} are TRUE, then the new and old alerts are declared the same and \textbf{Open};
\item If the alerts have the same group id, then they are mapped to the same line and the kind of block is the same. If it´s the same rule, and one of the features about the method is the same, we consider that they are the same alert. So if \textbf{Same Rule} is TRUE and \textbf{Same Group Id} is TRUE and \textbf{Same Method Code} or \textbf{Same Method Group} or \textbf{Same method Name} are TRUE, then the new and old alerts are declared the same and \textbf{Open};
\item if the alerts do not have the same group id, then we must have evidence that the method is the same and the line distance must be less then 5 lines, and the rule must be the same. So if \textbf{Same Rule} is TRUE and \textbf{Same Group Id} is TRUE and \textbf{Same Method Code} or \textbf{Same Method Group} or \textbf{Same method Name} are TRUE and \textbf{Line Distance} < 5, then then the new and old alerts are declared the same and \textbf{Open};
\item In all the other cases, the alerts are declared not the same and. If they do not match with any of the alerts of the other version, then are declared \textbf{Fixed} or \textbf{New}, depending on the version where they are located.
        
\end{itemize}

\section{Comparing new alerts with new SATD comments}\label{results}

In this section, we select some tagged versions of the project ArgoUML.
For each pair of sequential versions, we generate the PMD Alerts and
categorise them in New, Fixed and Open using the algorithm described in
Section \ref{alg}. We want to understand if the amount of new alerts,
normalized by the magnitude of the overall change between two versions,
is a good proxy for the amount of kludge introduced in the code base. A
first approach we try in this preliminary investigation is to measure
the correlation between the normalized amount of new alerts and comments
that indicate Self Admitted Technical Debt.

In Potdar and Shihab (2014), the authors discuss that the existence of
comments that contain some specific patterns may indicate what they call
Self-Admitted Technical Debts (SATD). In Sierra, Shihab, and Kamei
(2019), Self-Admitted Technical Debt is defined as the event in which
the developer consciously introduce debt. According to these two works,
the developer acknowledges the SATD in the form of comments. In Wehaibi
(2016) we can find some patterns based on the work of Potdar and Shihab.
For instance, some of these patterns are ``hack'', ``retarded'',
``remove this code'', ``treat this as a soft error'', ``kludge'',
``fixme'', ``this isn't quite right'', ``fix this crap'', ``abandon all
hope'' and ``kaboom''.

In this document, we selected some versions from the project ArgoUML and
extracted the PMD alerts, using the the procedure described in Section
\ref{alg}. Figure \ref{timeseries} shows three metrics related to the
transitions of versions.

In the first plot, ``Change'' tries to measure the amount of difference
between versions, summing the module of the difference between the
number of lines of code of new and old files:

\begin{equation} \label{eq_change} \sum_{f \in files}{|\#LoC_{f, new} - \#LoC_{f, old}|} \end{equation}

The second plot shows the number of new and fixed alerts, as categorized
by the algorithm in Section \ref{alg}.

The third plot shows the number of comments that contain expressions
listed in Wehaibi (2016). We categorize each comment as new, fixed and
open using a simple approach: if the text in the comment is the same,
the comments are classified as ``open'', the other ones are classified
as fixed, if they are in the old version and open if they are in the new
version.

\small

\normalsize

\small

\begin{figure}
\centering
\includegraphics{report_files/figure-latex/unnamed-chunk-20-1.pdf}
\caption{\label{timeseries}Changes, alerts and comments}
\end{figure}

\normalsize

\newpage

We try to measure if there is a correlation between the amount of new
alerts and the amount of new comments.

Figure \ref{scatter_prop} shows the relation between the proportion of
new alerts and the proportion of new comments:

\[PropNewAlerts = \frac{NewAlerts}{NewAlerts + OldAlerts}\]

\[PropNewComments = \frac{NewComments}{NewComments + OldComments}\]

We can see that there is a positive correlation.

\small

\begin{figure}
\centering
\includegraphics{report_files/figure-latex/unnamed-chunk-21-1.pdf}
\caption{\label{scatter_prop}Proportion of new alerts x Proportion of
new comments}
\end{figure}

\normalsize

In Figure \ref{scatter_diff} we correlate two metrics based on the
difference between the number of alerts and comments normalized by the
amount of change:

\[DiffNewAlerts = \frac{NewAlerts - OldAlerts}{Change}\]

\[DiffNewComments = \frac{NewComments - OldComments}{Change}\]

Where Change is calculated as in Equation \ref{eq_change}.

We can see that there is a positive correlation too.

\small

\begin{figure}
\centering
\includegraphics{report_files/figure-latex/unnamed-chunk-22-1.pdf}
\caption{\label{scatter_diff}Proportion of new alerts x Proportion of
new comments}
\end{figure}

\normalsize

It´s necessary to verify if the positive correlation that we found is
statistically significant. Table \ref{tab_reg} shows the results of
these regressions. In the form:

\[ NewAlertsProportion = \alpha + \beta NewCommentsProportions \]

and

\[ AlertsNormalizedDifferences = \alpha + \beta AlertsNormalizedDifferences \]

As we can see by the P-Value of the betas, we cannot reject the null
hypothesis in which there is no relation between comments and alerts.

\small

\begin{table}

\caption{\label{tab:unnamed-chunk-23}\label{tab_reg} Regression: alerts on comments}
\centering
\begin{tabular}[t]{l|l|l|l|l|l|l}
\hline
\multicolumn{1}{c|}{ } & \multicolumn{3}{c|}{Alerts Norm. Differences} & \multicolumn{3}{c}{New Alerts Proportions} \\
\cline{2-4} \cline{5-7}
Characteristic & Beta & 95\% CI & p-value & Beta & 95\% CI & p-value\\
\hline
(Intercept) & 0.34 & 0.17, 0.50 & <0.001 & -0.02 & -0.04, 0.00 & 0.045\\
\hline
New Comments Proportion & 0.22 & -0.01, 0.46 & 0.058 &  &  & \\
\hline
Comments Norm. Difference &  &  &  & 6.5 & -2.6, 16 & 0.16\\
\hline
\multicolumn{7}{l}{\textsuperscript{1} CI = Confidence Interval, CI = Confidence Interval}\\
\end{tabular}
\end{table}

\normalsize

The sign of the betas are positive as expected but maybe we did not achieve the necessary power in the test in order to be able to reject the null hypothesis and accept that thereis a correlation between comments and alerts. A possible next step is to run the same procedures for more versions of the project and for more projects. Another possible next step, which is more difficult but more promising is to refine the way we select the comments. There are papers that
use more sofisticated schemes to identify SATD comments. 

\section{References}

\hypertarget{refs}{}
\leavevmode\hypertarget{ref-Potdar2014}{}%
Potdar, Aniket, and Emad Shihab. 2014. ``An exploratory study on
self-admitted technical debt.'' \emph{Proceedings - 30th International
Conference on Software Maintenance and Evolution, ICSME 2014}, 91--100.
\url{https://doi.org/10.1109/ICSME.2014.31}.

\leavevmode\hypertarget{ref-Sierra2019}{}%
Sierra, Giancarlo, Emad Shihab, and Yasutaka Kamei. 2019. ``A survey of
self-admitted technical debt.'' Elsevier Inc.
\url{https://doi.org/10.1016/j.jss.2019.02.056}.

\leavevmode\hypertarget{ref-Wehaibi2016}{}%
Wehaibi, Sultan. 2016. ``Satd-Patterns.''
\url{https://github.com/xsultan/satd-patterns}.

\end{document}
